\documentclass[11pt]{article}

    \usepackage[breakable]{tcolorbox}
    \usepackage{parskip} % Stop auto-indenting (to mimic markdown behaviour)
    

    % Basic figure setup, for now with no caption control since it's done
    % automatically by Pandoc (which extracts ![](path) syntax from Markdown).
    \usepackage{graphicx}
    % Maintain compatibility with old templates. Remove in nbconvert 6.0
    \let\Oldincludegraphics\includegraphics
    % Ensure that by default, figures have no caption (until we provide a
    % proper Figure object with a Caption API and a way to capture that
    % in the conversion process - todo).
    \usepackage{caption}
    \DeclareCaptionFormat{nocaption}{}
    \captionsetup{format=nocaption,aboveskip=0pt,belowskip=0pt}

    \usepackage{float}
    \floatplacement{figure}{H} % forces figures to be placed at the correct location
    \usepackage{xcolor} % Allow colors to be defined
    \usepackage{enumerate} % Needed for markdown enumerations to work
    \usepackage{geometry} % Used to adjust the document margins
    \usepackage{amsmath} % Equations
    \usepackage{amssymb} % Equations
    \usepackage{textcomp} % defines textquotesingle
    % Hack from http://tex.stackexchange.com/a/47451/13684:
    \AtBeginDocument{%
        \def\PYZsq{\textquotesingle}% Upright quotes in Pygmentized code
    }
    \usepackage{upquote} % Upright quotes for verbatim code
    \usepackage{eurosym} % defines \euro

    \usepackage{iftex}
    \ifPDFTeX
        \usepackage[T1]{fontenc}
        \IfFileExists{alphabeta.sty}{
              \usepackage{alphabeta}
          }{
              \usepackage[mathletters]{ucs}
              \usepackage[utf8x]{inputenc}
          }
    \else
        \usepackage{fontspec}
        \usepackage{unicode-math}
    \fi

    \usepackage{fancyvrb} % verbatim replacement that allows latex
    \usepackage{grffile} % extends the file name processing of package graphics
                         % to support a larger range
    \makeatletter % fix for old versions of grffile with XeLaTeX
    \@ifpackagelater{grffile}{2019/11/01}
    {
      % Do nothing on new versions
    }
    {
      \def\Gread@@xetex#1{%
        \IfFileExists{"\Gin@base".bb}%
        {\Gread@eps{\Gin@base.bb}}%
        {\Gread@@xetex@aux#1}%
      }
    }
    \makeatother
    \usepackage[Export]{adjustbox} % Used to constrain images to a maximum size
    \adjustboxset{max size={0.9\linewidth}{0.9\paperheight}}

    % The hyperref package gives us a pdf with properly built
    % internal navigation ('pdf bookmarks' for the table of contents,
    % internal cross-reference links, web links for URLs, etc.)
    \usepackage{hyperref}
    % The default LaTeX title has an obnoxious amount of whitespace. By default,
    % titling removes some of it. It also provides customization options.
    \usepackage{titling}
    \usepackage{longtable} % longtable support required by pandoc >1.10
    \usepackage{booktabs}  % table support for pandoc > 1.12.2
    \usepackage{array}     % table support for pandoc >= 2.11.3
    \usepackage{calc}      % table minipage width calculation for pandoc >= 2.11.1
    \usepackage[inline]{enumitem} % IRkernel/repr support (it uses the enumerate* environment)
    \usepackage[normalem]{ulem} % ulem is needed to support strikethroughs (\sout)
                                % normalem makes italics be italics, not underlines
    \usepackage{mathrsfs}
    

    
    % Colors for the hyperref package
    \definecolor{urlcolor}{rgb}{0,.145,.698}
    \definecolor{linkcolor}{rgb}{.71,0.21,0.01}
    \definecolor{citecolor}{rgb}{.12,.54,.11}

    % ANSI colors
    \definecolor{ansi-black}{HTML}{3E424D}
    \definecolor{ansi-black-intense}{HTML}{282C36}
    \definecolor{ansi-red}{HTML}{E75C58}
    \definecolor{ansi-red-intense}{HTML}{B22B31}
    \definecolor{ansi-green}{HTML}{00A250}
    \definecolor{ansi-green-intense}{HTML}{007427}
    \definecolor{ansi-yellow}{HTML}{DDB62B}
    \definecolor{ansi-yellow-intense}{HTML}{B27D12}
    \definecolor{ansi-blue}{HTML}{208FFB}
    \definecolor{ansi-blue-intense}{HTML}{0065CA}
    \definecolor{ansi-magenta}{HTML}{D160C4}
    \definecolor{ansi-magenta-intense}{HTML}{A03196}
    \definecolor{ansi-cyan}{HTML}{60C6C8}
    \definecolor{ansi-cyan-intense}{HTML}{258F8F}
    \definecolor{ansi-white}{HTML}{C5C1B4}
    \definecolor{ansi-white-intense}{HTML}{A1A6B2}
    \definecolor{ansi-default-inverse-fg}{HTML}{FFFFFF}
    \definecolor{ansi-default-inverse-bg}{HTML}{000000}

    % common color for the border for error outputs.
    \definecolor{outerrorbackground}{HTML}{FFDFDF}

    % commands and environments needed by pandoc snippets
    % extracted from the output of `pandoc -s`
    \providecommand{\tightlist}{%
      \setlength{\itemsep}{0pt}\setlength{\parskip}{0pt}}
    \DefineVerbatimEnvironment{Highlighting}{Verbatim}{commandchars=\\\{\}}
    % Add ',fontsize=\small' for more characters per line
    \newenvironment{Shaded}{}{}
    \newcommand{\KeywordTok}[1]{\textcolor[rgb]{0.00,0.44,0.13}{\textbf{{#1}}}}
    \newcommand{\DataTypeTok}[1]{\textcolor[rgb]{0.56,0.13,0.00}{{#1}}}
    \newcommand{\DecValTok}[1]{\textcolor[rgb]{0.25,0.63,0.44}{{#1}}}
    \newcommand{\BaseNTok}[1]{\textcolor[rgb]{0.25,0.63,0.44}{{#1}}}
    \newcommand{\FloatTok}[1]{\textcolor[rgb]{0.25,0.63,0.44}{{#1}}}
    \newcommand{\CharTok}[1]{\textcolor[rgb]{0.25,0.44,0.63}{{#1}}}
    \newcommand{\StringTok}[1]{\textcolor[rgb]{0.25,0.44,0.63}{{#1}}}
    \newcommand{\CommentTok}[1]{\textcolor[rgb]{0.38,0.63,0.69}{\textit{{#1}}}}
    \newcommand{\OtherTok}[1]{\textcolor[rgb]{0.00,0.44,0.13}{{#1}}}
    \newcommand{\AlertTok}[1]{\textcolor[rgb]{1.00,0.00,0.00}{\textbf{{#1}}}}
    \newcommand{\FunctionTok}[1]{\textcolor[rgb]{0.02,0.16,0.49}{{#1}}}
    \newcommand{\RegionMarkerTok}[1]{{#1}}
    \newcommand{\ErrorTok}[1]{\textcolor[rgb]{1.00,0.00,0.00}{\textbf{{#1}}}}
    \newcommand{\NormalTok}[1]{{#1}}

    % Additional commands for more recent versions of Pandoc
    \newcommand{\ConstantTok}[1]{\textcolor[rgb]{0.53,0.00,0.00}{{#1}}}
    \newcommand{\SpecialCharTok}[1]{\textcolor[rgb]{0.25,0.44,0.63}{{#1}}}
    \newcommand{\VerbatimStringTok}[1]{\textcolor[rgb]{0.25,0.44,0.63}{{#1}}}
    \newcommand{\SpecialStringTok}[1]{\textcolor[rgb]{0.73,0.40,0.53}{{#1}}}
    \newcommand{\ImportTok}[1]{{#1}}
    \newcommand{\DocumentationTok}[1]{\textcolor[rgb]{0.73,0.13,0.13}{\textit{{#1}}}}
    \newcommand{\AnnotationTok}[1]{\textcolor[rgb]{0.38,0.63,0.69}{\textbf{\textit{{#1}}}}}
    \newcommand{\CommentVarTok}[1]{\textcolor[rgb]{0.38,0.63,0.69}{\textbf{\textit{{#1}}}}}
    \newcommand{\VariableTok}[1]{\textcolor[rgb]{0.10,0.09,0.49}{{#1}}}
    \newcommand{\ControlFlowTok}[1]{\textcolor[rgb]{0.00,0.44,0.13}{\textbf{{#1}}}}
    \newcommand{\OperatorTok}[1]{\textcolor[rgb]{0.40,0.40,0.40}{{#1}}}
    \newcommand{\BuiltInTok}[1]{{#1}}
    \newcommand{\ExtensionTok}[1]{{#1}}
    \newcommand{\PreprocessorTok}[1]{\textcolor[rgb]{0.74,0.48,0.00}{{#1}}}
    \newcommand{\AttributeTok}[1]{\textcolor[rgb]{0.49,0.56,0.16}{{#1}}}
    \newcommand{\InformationTok}[1]{\textcolor[rgb]{0.38,0.63,0.69}{\textbf{\textit{{#1}}}}}
    \newcommand{\WarningTok}[1]{\textcolor[rgb]{0.38,0.63,0.69}{\textbf{\textit{{#1}}}}}


    % Define a nice break command that doesn't care if a line doesn't already
    % exist.
    \def\br{\hspace*{\fill} \\* }
    % Math Jax compatibility definitions
    \def\gt{>}
    \def\lt{<}
    \let\Oldtex\TeX
    \let\Oldlatex\LaTeX
    \renewcommand{\TeX}{\textrm{\Oldtex}}
    \renewcommand{\LaTeX}{\textrm{\Oldlatex}}
    % Document parameters
    % Document title
    \title{ChandonnetModule06Lab01}
    
    
    
    
    
% Pygments definitions
\makeatletter
\def\PY@reset{\let\PY@it=\relax \let\PY@bf=\relax%
    \let\PY@ul=\relax \let\PY@tc=\relax%
    \let\PY@bc=\relax \let\PY@ff=\relax}
\def\PY@tok#1{\csname PY@tok@#1\endcsname}
\def\PY@toks#1+{\ifx\relax#1\empty\else%
    \PY@tok{#1}\expandafter\PY@toks\fi}
\def\PY@do#1{\PY@bc{\PY@tc{\PY@ul{%
    \PY@it{\PY@bf{\PY@ff{#1}}}}}}}
\def\PY#1#2{\PY@reset\PY@toks#1+\relax+\PY@do{#2}}

\@namedef{PY@tok@w}{\def\PY@tc##1{\textcolor[rgb]{0.73,0.73,0.73}{##1}}}
\@namedef{PY@tok@c}{\let\PY@it=\textit\def\PY@tc##1{\textcolor[rgb]{0.24,0.48,0.48}{##1}}}
\@namedef{PY@tok@cp}{\def\PY@tc##1{\textcolor[rgb]{0.61,0.40,0.00}{##1}}}
\@namedef{PY@tok@k}{\let\PY@bf=\textbf\def\PY@tc##1{\textcolor[rgb]{0.00,0.50,0.00}{##1}}}
\@namedef{PY@tok@kp}{\def\PY@tc##1{\textcolor[rgb]{0.00,0.50,0.00}{##1}}}
\@namedef{PY@tok@kt}{\def\PY@tc##1{\textcolor[rgb]{0.69,0.00,0.25}{##1}}}
\@namedef{PY@tok@o}{\def\PY@tc##1{\textcolor[rgb]{0.40,0.40,0.40}{##1}}}
\@namedef{PY@tok@ow}{\let\PY@bf=\textbf\def\PY@tc##1{\textcolor[rgb]{0.67,0.13,1.00}{##1}}}
\@namedef{PY@tok@nb}{\def\PY@tc##1{\textcolor[rgb]{0.00,0.50,0.00}{##1}}}
\@namedef{PY@tok@nf}{\def\PY@tc##1{\textcolor[rgb]{0.00,0.00,1.00}{##1}}}
\@namedef{PY@tok@nc}{\let\PY@bf=\textbf\def\PY@tc##1{\textcolor[rgb]{0.00,0.00,1.00}{##1}}}
\@namedef{PY@tok@nn}{\let\PY@bf=\textbf\def\PY@tc##1{\textcolor[rgb]{0.00,0.00,1.00}{##1}}}
\@namedef{PY@tok@ne}{\let\PY@bf=\textbf\def\PY@tc##1{\textcolor[rgb]{0.80,0.25,0.22}{##1}}}
\@namedef{PY@tok@nv}{\def\PY@tc##1{\textcolor[rgb]{0.10,0.09,0.49}{##1}}}
\@namedef{PY@tok@no}{\def\PY@tc##1{\textcolor[rgb]{0.53,0.00,0.00}{##1}}}
\@namedef{PY@tok@nl}{\def\PY@tc##1{\textcolor[rgb]{0.46,0.46,0.00}{##1}}}
\@namedef{PY@tok@ni}{\let\PY@bf=\textbf\def\PY@tc##1{\textcolor[rgb]{0.44,0.44,0.44}{##1}}}
\@namedef{PY@tok@na}{\def\PY@tc##1{\textcolor[rgb]{0.41,0.47,0.13}{##1}}}
\@namedef{PY@tok@nt}{\let\PY@bf=\textbf\def\PY@tc##1{\textcolor[rgb]{0.00,0.50,0.00}{##1}}}
\@namedef{PY@tok@nd}{\def\PY@tc##1{\textcolor[rgb]{0.67,0.13,1.00}{##1}}}
\@namedef{PY@tok@s}{\def\PY@tc##1{\textcolor[rgb]{0.73,0.13,0.13}{##1}}}
\@namedef{PY@tok@sd}{\let\PY@it=\textit\def\PY@tc##1{\textcolor[rgb]{0.73,0.13,0.13}{##1}}}
\@namedef{PY@tok@si}{\let\PY@bf=\textbf\def\PY@tc##1{\textcolor[rgb]{0.64,0.35,0.47}{##1}}}
\@namedef{PY@tok@se}{\let\PY@bf=\textbf\def\PY@tc##1{\textcolor[rgb]{0.67,0.36,0.12}{##1}}}
\@namedef{PY@tok@sr}{\def\PY@tc##1{\textcolor[rgb]{0.64,0.35,0.47}{##1}}}
\@namedef{PY@tok@ss}{\def\PY@tc##1{\textcolor[rgb]{0.10,0.09,0.49}{##1}}}
\@namedef{PY@tok@sx}{\def\PY@tc##1{\textcolor[rgb]{0.00,0.50,0.00}{##1}}}
\@namedef{PY@tok@m}{\def\PY@tc##1{\textcolor[rgb]{0.40,0.40,0.40}{##1}}}
\@namedef{PY@tok@gh}{\let\PY@bf=\textbf\def\PY@tc##1{\textcolor[rgb]{0.00,0.00,0.50}{##1}}}
\@namedef{PY@tok@gu}{\let\PY@bf=\textbf\def\PY@tc##1{\textcolor[rgb]{0.50,0.00,0.50}{##1}}}
\@namedef{PY@tok@gd}{\def\PY@tc##1{\textcolor[rgb]{0.63,0.00,0.00}{##1}}}
\@namedef{PY@tok@gi}{\def\PY@tc##1{\textcolor[rgb]{0.00,0.52,0.00}{##1}}}
\@namedef{PY@tok@gr}{\def\PY@tc##1{\textcolor[rgb]{0.89,0.00,0.00}{##1}}}
\@namedef{PY@tok@ge}{\let\PY@it=\textit}
\@namedef{PY@tok@gs}{\let\PY@bf=\textbf}
\@namedef{PY@tok@gp}{\let\PY@bf=\textbf\def\PY@tc##1{\textcolor[rgb]{0.00,0.00,0.50}{##1}}}
\@namedef{PY@tok@go}{\def\PY@tc##1{\textcolor[rgb]{0.44,0.44,0.44}{##1}}}
\@namedef{PY@tok@gt}{\def\PY@tc##1{\textcolor[rgb]{0.00,0.27,0.87}{##1}}}
\@namedef{PY@tok@err}{\def\PY@bc##1{{\setlength{\fboxsep}{\string -\fboxrule}\fcolorbox[rgb]{1.00,0.00,0.00}{1,1,1}{\strut ##1}}}}
\@namedef{PY@tok@kc}{\let\PY@bf=\textbf\def\PY@tc##1{\textcolor[rgb]{0.00,0.50,0.00}{##1}}}
\@namedef{PY@tok@kd}{\let\PY@bf=\textbf\def\PY@tc##1{\textcolor[rgb]{0.00,0.50,0.00}{##1}}}
\@namedef{PY@tok@kn}{\let\PY@bf=\textbf\def\PY@tc##1{\textcolor[rgb]{0.00,0.50,0.00}{##1}}}
\@namedef{PY@tok@kr}{\let\PY@bf=\textbf\def\PY@tc##1{\textcolor[rgb]{0.00,0.50,0.00}{##1}}}
\@namedef{PY@tok@bp}{\def\PY@tc##1{\textcolor[rgb]{0.00,0.50,0.00}{##1}}}
\@namedef{PY@tok@fm}{\def\PY@tc##1{\textcolor[rgb]{0.00,0.00,1.00}{##1}}}
\@namedef{PY@tok@vc}{\def\PY@tc##1{\textcolor[rgb]{0.10,0.09,0.49}{##1}}}
\@namedef{PY@tok@vg}{\def\PY@tc##1{\textcolor[rgb]{0.10,0.09,0.49}{##1}}}
\@namedef{PY@tok@vi}{\def\PY@tc##1{\textcolor[rgb]{0.10,0.09,0.49}{##1}}}
\@namedef{PY@tok@vm}{\def\PY@tc##1{\textcolor[rgb]{0.10,0.09,0.49}{##1}}}
\@namedef{PY@tok@sa}{\def\PY@tc##1{\textcolor[rgb]{0.73,0.13,0.13}{##1}}}
\@namedef{PY@tok@sb}{\def\PY@tc##1{\textcolor[rgb]{0.73,0.13,0.13}{##1}}}
\@namedef{PY@tok@sc}{\def\PY@tc##1{\textcolor[rgb]{0.73,0.13,0.13}{##1}}}
\@namedef{PY@tok@dl}{\def\PY@tc##1{\textcolor[rgb]{0.73,0.13,0.13}{##1}}}
\@namedef{PY@tok@s2}{\def\PY@tc##1{\textcolor[rgb]{0.73,0.13,0.13}{##1}}}
\@namedef{PY@tok@sh}{\def\PY@tc##1{\textcolor[rgb]{0.73,0.13,0.13}{##1}}}
\@namedef{PY@tok@s1}{\def\PY@tc##1{\textcolor[rgb]{0.73,0.13,0.13}{##1}}}
\@namedef{PY@tok@mb}{\def\PY@tc##1{\textcolor[rgb]{0.40,0.40,0.40}{##1}}}
\@namedef{PY@tok@mf}{\def\PY@tc##1{\textcolor[rgb]{0.40,0.40,0.40}{##1}}}
\@namedef{PY@tok@mh}{\def\PY@tc##1{\textcolor[rgb]{0.40,0.40,0.40}{##1}}}
\@namedef{PY@tok@mi}{\def\PY@tc##1{\textcolor[rgb]{0.40,0.40,0.40}{##1}}}
\@namedef{PY@tok@il}{\def\PY@tc##1{\textcolor[rgb]{0.40,0.40,0.40}{##1}}}
\@namedef{PY@tok@mo}{\def\PY@tc##1{\textcolor[rgb]{0.40,0.40,0.40}{##1}}}
\@namedef{PY@tok@ch}{\let\PY@it=\textit\def\PY@tc##1{\textcolor[rgb]{0.24,0.48,0.48}{##1}}}
\@namedef{PY@tok@cm}{\let\PY@it=\textit\def\PY@tc##1{\textcolor[rgb]{0.24,0.48,0.48}{##1}}}
\@namedef{PY@tok@cpf}{\let\PY@it=\textit\def\PY@tc##1{\textcolor[rgb]{0.24,0.48,0.48}{##1}}}
\@namedef{PY@tok@c1}{\let\PY@it=\textit\def\PY@tc##1{\textcolor[rgb]{0.24,0.48,0.48}{##1}}}
\@namedef{PY@tok@cs}{\let\PY@it=\textit\def\PY@tc##1{\textcolor[rgb]{0.24,0.48,0.48}{##1}}}

\def\PYZbs{\char`\\}
\def\PYZus{\char`\_}
\def\PYZob{\char`\{}
\def\PYZcb{\char`\}}
\def\PYZca{\char`\^}
\def\PYZam{\char`\&}
\def\PYZlt{\char`\<}
\def\PYZgt{\char`\>}
\def\PYZsh{\char`\#}
\def\PYZpc{\char`\%}
\def\PYZdl{\char`\$}
\def\PYZhy{\char`\-}
\def\PYZsq{\char`\'}
\def\PYZdq{\char`\"}
\def\PYZti{\char`\~}
% for compatibility with earlier versions
\def\PYZat{@}
\def\PYZlb{[}
\def\PYZrb{]}
\makeatother


    % For linebreaks inside Verbatim environment from package fancyvrb.
    \makeatletter
        \newbox\Wrappedcontinuationbox
        \newbox\Wrappedvisiblespacebox
        \newcommand*\Wrappedvisiblespace {\textcolor{red}{\textvisiblespace}}
        \newcommand*\Wrappedcontinuationsymbol {\textcolor{red}{\llap{\tiny$\m@th\hookrightarrow$}}}
        \newcommand*\Wrappedcontinuationindent {3ex }
        \newcommand*\Wrappedafterbreak {\kern\Wrappedcontinuationindent\copy\Wrappedcontinuationbox}
        % Take advantage of the already applied Pygments mark-up to insert
        % potential linebreaks for TeX processing.
        %        {, <, #, %, $, ' and ": go to next line.
        %        _, }, ^, &, >, - and ~: stay at end of broken line.
        % Use of \textquotesingle for straight quote.
        \newcommand*\Wrappedbreaksatspecials {%
            \def\PYGZus{\discretionary{\char`\_}{\Wrappedafterbreak}{\char`\_}}%
            \def\PYGZob{\discretionary{}{\Wrappedafterbreak\char`\{}{\char`\{}}%
            \def\PYGZcb{\discretionary{\char`\}}{\Wrappedafterbreak}{\char`\}}}%
            \def\PYGZca{\discretionary{\char`\^}{\Wrappedafterbreak}{\char`\^}}%
            \def\PYGZam{\discretionary{\char`\&}{\Wrappedafterbreak}{\char`\&}}%
            \def\PYGZlt{\discretionary{}{\Wrappedafterbreak\char`\<}{\char`\<}}%
            \def\PYGZgt{\discretionary{\char`\>}{\Wrappedafterbreak}{\char`\>}}%
            \def\PYGZsh{\discretionary{}{\Wrappedafterbreak\char`\#}{\char`\#}}%
            \def\PYGZpc{\discretionary{}{\Wrappedafterbreak\char`\%}{\char`\%}}%
            \def\PYGZdl{\discretionary{}{\Wrappedafterbreak\char`\$}{\char`\$}}%
            \def\PYGZhy{\discretionary{\char`\-}{\Wrappedafterbreak}{\char`\-}}%
            \def\PYGZsq{\discretionary{}{\Wrappedafterbreak\textquotesingle}{\textquotesingle}}%
            \def\PYGZdq{\discretionary{}{\Wrappedafterbreak\char`\"}{\char`\"}}%
            \def\PYGZti{\discretionary{\char`\~}{\Wrappedafterbreak}{\char`\~}}%
        }
        % Some characters . , ; ? ! / are not pygmentized.
        % This macro makes them "active" and they will insert potential linebreaks
        \newcommand*\Wrappedbreaksatpunct {%
            \lccode`\~`\.\lowercase{\def~}{\discretionary{\hbox{\char`\.}}{\Wrappedafterbreak}{\hbox{\char`\.}}}%
            \lccode`\~`\,\lowercase{\def~}{\discretionary{\hbox{\char`\,}}{\Wrappedafterbreak}{\hbox{\char`\,}}}%
            \lccode`\~`\;\lowercase{\def~}{\discretionary{\hbox{\char`\;}}{\Wrappedafterbreak}{\hbox{\char`\;}}}%
            \lccode`\~`\:\lowercase{\def~}{\discretionary{\hbox{\char`\:}}{\Wrappedafterbreak}{\hbox{\char`\:}}}%
            \lccode`\~`\?\lowercase{\def~}{\discretionary{\hbox{\char`\?}}{\Wrappedafterbreak}{\hbox{\char`\?}}}%
            \lccode`\~`\!\lowercase{\def~}{\discretionary{\hbox{\char`\!}}{\Wrappedafterbreak}{\hbox{\char`\!}}}%
            \lccode`\~`\/\lowercase{\def~}{\discretionary{\hbox{\char`\/}}{\Wrappedafterbreak}{\hbox{\char`\/}}}%
            \catcode`\.\active
            \catcode`\,\active
            \catcode`\;\active
            \catcode`\:\active
            \catcode`\?\active
            \catcode`\!\active
            \catcode`\/\active
            \lccode`\~`\~
        }
    \makeatother

    \let\OriginalVerbatim=\Verbatim
    \makeatletter
    \renewcommand{\Verbatim}[1][1]{%
        %\parskip\z@skip
        \sbox\Wrappedcontinuationbox {\Wrappedcontinuationsymbol}%
        \sbox\Wrappedvisiblespacebox {\FV@SetupFont\Wrappedvisiblespace}%
        \def\FancyVerbFormatLine ##1{\hsize\linewidth
            \vtop{\raggedright\hyphenpenalty\z@\exhyphenpenalty\z@
                \doublehyphendemerits\z@\finalhyphendemerits\z@
                \strut ##1\strut}%
        }%
        % If the linebreak is at a space, the latter will be displayed as visible
        % space at end of first line, and a continuation symbol starts next line.
        % Stretch/shrink are however usually zero for typewriter font.
        \def\FV@Space {%
            \nobreak\hskip\z@ plus\fontdimen3\font minus\fontdimen4\font
            \discretionary{\copy\Wrappedvisiblespacebox}{\Wrappedafterbreak}
            {\kern\fontdimen2\font}%
        }%

        % Allow breaks at special characters using \PYG... macros.
        \Wrappedbreaksatspecials
        % Breaks at punctuation characters . , ; ? ! and / need catcode=\active
        \OriginalVerbatim[#1,codes*=\Wrappedbreaksatpunct]%
    }
    \makeatother

    % Exact colors from NB
    \definecolor{incolor}{HTML}{303F9F}
    \definecolor{outcolor}{HTML}{D84315}
    \definecolor{cellborder}{HTML}{CFCFCF}
    \definecolor{cellbackground}{HTML}{F7F7F7}

    % prompt
    \makeatletter
    \newcommand{\boxspacing}{\kern\kvtcb@left@rule\kern\kvtcb@boxsep}
    \makeatother
    \newcommand{\prompt}[4]{
        {\ttfamily\llap{{\color{#2}[#3]:\hspace{3pt}#4}}\vspace{-\baselineskip}}
    }
    

    
    % Prevent overflowing lines due to hard-to-break entities
    \sloppy
    % Setup hyperref package
    \hypersetup{
      breaklinks=true,  % so long urls are correctly broken across lines
      colorlinks=true,
      urlcolor=urlcolor,
      linkcolor=linkcolor,
      citecolor=citecolor,
      }
    % Slightly bigger margins than the latex defaults
    
    \geometry{verbose,tmargin=1in,bmargin=1in,lmargin=1in,rmargin=1in}
    
    

\begin{document}
    
    \maketitle
    
    

    
    \hypertarget{assignment-6}{%
\section{Assignment 6}\label{assignment-6}}

\hypertarget{ray-chandonnet}{%
\subsubsection{Ray Chandonnet}\label{ray-chandonnet}}

\hypertarget{section}{%
\subsubsection{12/4/2022}\label{section}}

    \hypertarget{question-1}{%
\subsection{Question 1}\label{question-1}}

\begin{enumerate}
\def\labelenumi{\arabic{enumi})}
\item
  import the random library.
\item
  Use \texttt{random.seed(10)} to initialize a pseudorandom number
  generator.
\item
  Create a list of 50 random integers from 0 to 15. Call this list
  \texttt{int\_list}.
\item
  Print the 10th and 30th elements of the list.
\end{enumerate}

You will need to use list comprehension to do this. The syntax for list
comprehension is: =
\texttt{{[}\textless{}expression\textgreater{}\ for\ \textless{}item\textgreater{}\ in\ \textless{}iterable\textgreater{}{]}}.
For this question your expression will be a randint generator from the
random library and your iterable will be \texttt{range()}. Researh the
documentation on how to use both functions.

    \begin{tcolorbox}[breakable, size=fbox, boxrule=1pt, pad at break*=1mm,colback=cellbackground, colframe=cellborder]
\prompt{In}{incolor}{1}{\boxspacing}
\begin{Verbatim}[commandchars=\\\{\}]
\PY{c+c1}{\PYZsh{} Problem 1}
\PY{k+kn}{import} \PY{n+nn}{random} \PY{k}{as} \PY{n+nn}{r}
\PY{n}{r}\PY{o}{.}\PY{n}{seed}\PY{p}{(}\PY{l+m+mi}{10}\PY{p}{)}
\PY{n}{int\PYZus{}list} \PY{o}{=} \PY{p}{[}\PY{n}{r}\PY{o}{.}\PY{n}{randint}\PY{p}{(}\PY{l+m+mi}{0}\PY{p}{,}\PY{l+m+mi}{15}\PY{p}{)} \PY{k}{for} \PY{n}{counter} \PY{o+ow}{in} \PY{n+nb}{range}\PY{p}{(}\PY{l+m+mi}{50}\PY{p}{)}\PY{p}{]}
\PY{n+nb}{print}\PY{p}{(}\PY{l+s+s2}{\PYZdq{}}\PY{l+s+s2}{The entire list is :}\PY{l+s+s2}{\PYZdq{}}\PY{p}{,}\PY{n}{int\PYZus{}list}\PY{p}{)}
\PY{n+nb}{print}\PY{p}{(}\PY{l+s+s2}{\PYZdq{}}\PY{l+s+s2}{The 10th element is : }\PY{l+s+s2}{\PYZdq{}}\PY{p}{,} \PY{n}{int\PYZus{}list}\PY{p}{[}\PY{l+m+mi}{9}\PY{p}{]}\PY{p}{)}
\PY{n+nb}{print}\PY{p}{(}\PY{l+s+s2}{\PYZdq{}}\PY{l+s+s2}{The 29th element is : }\PY{l+s+s2}{\PYZdq{}}\PY{p}{,}\PY{n}{int\PYZus{}list}\PY{p}{[}\PY{l+m+mi}{29}\PY{p}{]}\PY{p}{)}
\PY{c+c1}{\PYZsh{} Note that since Python uses zero\PYZhy{}based indexing, the 10th element is at index \PYZsh{}9 and the 30th element is at index \PYZsh{}29}
\end{Verbatim}
\end{tcolorbox}

    \begin{Verbatim}[commandchars=\\\{\}]
The entire list is : [1, 13, 15, 0, 6, 14, 15, 8, 5, 1, 15, 10, 2, 7, 11, 1, 13,
4, 11, 12, 13, 9, 8, 14, 5, 9, 11, 4, 14, 7, 14, 12, 1, 0, 7, 4, 6, 9, 11, 7,
10, 14, 13, 15, 2, 10, 5, 7, 13, 7]
The 10th element is :  1
The 29th element is :  7
    \end{Verbatim}

    \hypertarget{question-2}{%
\subsection{Question 2}\label{question-2}}

\begin{enumerate}
\def\labelenumi{\arabic{enumi})}
\item
  import the string library.
\item
  Create the string \texttt{az\_upper} using
  \texttt{string.ascii\_uppercase}. This is a single string of uppercase
  letters
\item
  Create a list of each individual letter from the string. To do this
  you will need to iterate over the string and append each letter to the
  an empty list. Call this list \texttt{az\_list}.
\item
  Print the list.
\end{enumerate}

You will need to use a for-loop for this. The syntax for this for-loop
should be:

\texttt{for\ i\ in\ string\textgreater{}:\ \ \ \ \textless{}list\ operation\textgreater{}}

    \begin{tcolorbox}[breakable, size=fbox, boxrule=1pt, pad at break*=1mm,colback=cellbackground, colframe=cellborder]
\prompt{In}{incolor}{2}{\boxspacing}
\begin{Verbatim}[commandchars=\\\{\}]
\PY{c+c1}{\PYZsh{} Problem 2}
\PY{k+kn}{import} \PY{n+nn}{string} \PY{k}{as} \PY{n+nn}{s}
\PY{n}{az\PYZus{}upper} \PY{o}{=} \PY{n}{s}\PY{o}{.}\PY{n}{ascii\PYZus{}uppercase}
\PY{n+nb}{print}\PY{p}{(}\PY{n}{az\PYZus{}upper}\PY{p}{)}
\PY{n}{az\PYZus{}list}\PY{o}{=}\PY{p}{[}\PY{p}{]}
\PY{k}{for} \PY{n}{counter} \PY{o+ow}{in} \PY{n}{az\PYZus{}upper}\PY{p}{:}
    \PY{n}{az\PYZus{}list}\PY{o}{.}\PY{n}{append}\PY{p}{(}\PY{n}{counter}\PY{p}{)}
\PY{n+nb}{print}\PY{p}{(}\PY{n}{az\PYZus{}list}\PY{p}{)}
\end{Verbatim}
\end{tcolorbox}

    \begin{Verbatim}[commandchars=\\\{\}]
ABCDEFGHIJKLMNOPQRSTUVWXYZ
['A', 'B', 'C', 'D', 'E', 'F', 'G', 'H', 'I', 'J', 'K', 'L', 'M', 'N', 'O', 'P',
'Q', 'R', 'S', 'T', 'U', 'V', 'W', 'X', 'Y', 'Z']
    \end{Verbatim}

    \hypertarget{question-3}{%
\subsection{Question 3}\label{question-3}}

\begin{enumerate}
\def\labelenumi{\arabic{enumi})}
\item
  Create a set from 1 to 5. Call this \texttt{set\_1}.
\item
  Create a set from int\_list. Call this \texttt{set\_2}.
\item
  Create a set by finding the \texttt{symmetric\_difference()} of
  \texttt{set\_1} and \texttt{set\_2}. Call this \texttt{set\_3}.
\item
  What is the length of all three sets?
\end{enumerate}

    \begin{tcolorbox}[breakable, size=fbox, boxrule=1pt, pad at break*=1mm,colback=cellbackground, colframe=cellborder]
\prompt{In}{incolor}{3}{\boxspacing}
\begin{Verbatim}[commandchars=\\\{\}]
\PY{n}{set\PYZus{}1}\PY{o}{=}\PY{n+nb}{set}\PY{p}{(}\PY{n+nb}{range}\PY{p}{(}\PY{l+m+mi}{1}\PY{p}{,}\PY{l+m+mi}{6}\PY{p}{)}\PY{p}{)}
\PY{n+nb}{print}\PY{p}{(}\PY{l+s+s2}{\PYZdq{}}\PY{l+s+s2}{Set\PYZus{}1:}\PY{l+s+s2}{\PYZdq{}}\PY{p}{,}\PY{n}{set\PYZus{}1}\PY{p}{)}
\PY{n}{set\PYZus{}2}\PY{o}{=}\PY{n+nb}{set}\PY{p}{(}\PY{n}{int\PYZus{}list}\PY{p}{)}
\PY{n+nb}{print}\PY{p}{(}\PY{l+s+s2}{\PYZdq{}}\PY{l+s+s2}{Set\PYZus{}2:}\PY{l+s+s2}{\PYZdq{}}\PY{p}{,}\PY{n}{set\PYZus{}2}\PY{p}{)}
\PY{n}{set\PYZus{}3} \PY{o}{=} \PY{n}{set\PYZus{}1}\PY{o}{.}\PY{n}{symmetric\PYZus{}difference}\PY{p}{(}\PY{n}{set\PYZus{}2}\PY{p}{)}
\PY{n+nb}{print}\PY{p}{(}\PY{l+s+s2}{\PYZdq{}}\PY{l+s+s2}{Set\PYZus{}3:}\PY{l+s+s2}{\PYZdq{}}\PY{p}{,}\PY{n}{set\PYZus{}3}\PY{p}{)}
\PY{n}{len\PYZus{}1}\PY{o}{=}\PY{n+nb}{len}\PY{p}{(}\PY{n}{set\PYZus{}1}\PY{p}{)}
\PY{n}{len\PYZus{}2}\PY{o}{=}\PY{n+nb}{len}\PY{p}{(}\PY{n}{set\PYZus{}2}\PY{p}{)}
\PY{n}{len\PYZus{}3}\PY{o}{=}\PY{n+nb}{len}\PY{p}{(}\PY{n}{set\PYZus{}3}\PY{p}{)}
\PY{n+nb}{print}\PY{p}{(}\PY{l+s+s2}{\PYZdq{}}\PY{l+s+s2}{Set 1 has}\PY{l+s+s2}{\PYZdq{}}\PY{p}{,}\PY{n}{len\PYZus{}1}\PY{p}{,}\PY{l+s+s2}{\PYZdq{}}\PY{l+s+s2}{elements}\PY{l+s+s2}{\PYZdq{}}\PY{p}{)}
\PY{n+nb}{print}\PY{p}{(}\PY{l+s+s2}{\PYZdq{}}\PY{l+s+s2}{Set 2 has}\PY{l+s+s2}{\PYZdq{}}\PY{p}{,}\PY{n}{len\PYZus{}2}\PY{p}{,}\PY{l+s+s2}{\PYZdq{}}\PY{l+s+s2}{elements}\PY{l+s+s2}{\PYZdq{}}\PY{p}{)}
\PY{n+nb}{print}\PY{p}{(}\PY{l+s+s2}{\PYZdq{}}\PY{l+s+s2}{Set 3, the symmetric\PYZus{}difference set, has}\PY{l+s+s2}{\PYZdq{}}\PY{p}{,}\PY{n}{len\PYZus{}3}\PY{p}{,}\PY{l+s+s2}{\PYZdq{}}\PY{l+s+s2}{elements}\PY{l+s+s2}{\PYZdq{}}\PY{p}{)}
\PY{n+nb}{print}\PY{p}{(}\PY{l+s+s2}{\PYZdq{}}\PY{l+s+s2}{For my own comprehension \PYZhy{} that is the number of unique elements that do NOT appear in both sets (opposite of}\PY{l+s+s2}{\PYZdq{}}\PY{p}{)}
\PY{n+nb}{print}\PY{p}{(}\PY{l+s+s2}{\PYZdq{}}\PY{l+s+s2}{intersection)}\PY{l+s+s2}{\PYZdq{}}\PY{p}{)}
\PY{n}{set\PYZus{}4} \PY{o}{=} \PY{n}{set\PYZus{}1}\PY{o}{.}\PY{n}{intersection}\PY{p}{(}\PY{n}{set\PYZus{}2}\PY{p}{)}
\PY{n}{len\PYZus{}4}\PY{o}{=}\PY{n+nb}{len}\PY{p}{(}\PY{n}{set\PYZus{}4}\PY{p}{)}
\PY{n+nb}{print}\PY{p}{(}\PY{l+s+s2}{\PYZdq{}}\PY{l+s+s2}{Set 4, the intersection set, has}\PY{l+s+s2}{\PYZdq{}}\PY{p}{,}\PY{n}{len\PYZus{}4}\PY{p}{,}\PY{l+s+s2}{\PYZdq{}}\PY{l+s+s2}{elements}\PY{l+s+s2}{\PYZdq{}}\PY{p}{)}
\PY{n}{tot\PYZus{}elements}\PY{o}{=}\PY{n}{len\PYZus{}1}\PY{o}{+}\PY{n}{len\PYZus{}2}
\PY{n+nb}{print}\PY{p}{(}\PY{l+s+s2}{\PYZdq{}}\PY{l+s+s2}{There are a total of}\PY{l+s+s2}{\PYZdq{}}\PY{p}{,}\PY{n}{tot\PYZus{}elements}\PY{p}{,}\PY{l+s+s2}{\PYZdq{}}\PY{l+s+s2}{elements in the two sets}\PY{l+s+s2}{\PYZdq{}}\PY{p}{)}
\PY{n+nb}{print}\PY{p}{(}\PY{n}{len\PYZus{}4}\PY{p}{,}\PY{l+s+s2}{\PYZdq{}}\PY{l+s+s2}{of those elements appear both times = }\PY{l+s+s2}{\PYZdq{}}\PY{p}{,}\PY{n}{len\PYZus{}4}\PY{o}{*}\PY{l+m+mi}{2}\PY{p}{,}\PY{l+s+s2}{\PYZdq{}}\PY{l+s+s2}{instances}\PY{l+s+s2}{\PYZdq{}}\PY{p}{)}
\PY{n+nb}{print}\PY{p}{(}\PY{n}{len\PYZus{}3}\PY{p}{,}\PY{l+s+s2}{\PYZdq{}}\PY{l+s+s2}{of those elements only appear once = }\PY{l+s+s2}{\PYZdq{}}\PY{p}{,}\PY{n}{len\PYZus{}3}\PY{p}{,}\PY{l+s+s2}{\PYZdq{}}\PY{l+s+s2}{instances}\PY{l+s+s2}{\PYZdq{}}\PY{p}{)}
\PY{n+nb}{print}\PY{p}{(}\PY{l+s+s2}{\PYZdq{}}\PY{l+s+s2}{Added together, they equate to the total \PYZsh{} of elements:}\PY{l+s+s2}{\PYZdq{}}\PY{p}{,} \PY{n}{tot\PYZus{}elements}\PY{p}{,}\PY{l+s+s2}{\PYZdq{}}\PY{l+s+s2}{=}\PY{l+s+s2}{\PYZdq{}}\PY{p}{,}\PY{n}{len\PYZus{}4}\PY{o}{*}\PY{l+m+mi}{2}\PY{o}{+}\PY{n}{len\PYZus{}3}\PY{p}{)}
\end{Verbatim}
\end{tcolorbox}

    \begin{Verbatim}[commandchars=\\\{\}]
Set\_1: \{1, 2, 3, 4, 5\}
Set\_2: \{0, 1, 2, 4, 5, 6, 7, 8, 9, 10, 11, 12, 13, 14, 15\}
Set\_3: \{0, 3, 6, 7, 8, 9, 10, 11, 12, 13, 14, 15\}
Set 1 has 5 elements
Set 2 has 15 elements
Set 3, the symmetric\_difference set, has 12 elements
For my own comprehension - that is the number of unique elements that do NOT
appear in both sets (opposite of
intersection)
Set 4, the intersection set, has 4 elements
There are a total of 20 elements in the two sets
4 of those elements appear both times =  8 instances
12 of those elements only appear once =  12 instances
Added together, they equate to the total \# of elements: 20 = 20
    \end{Verbatim}

    \hypertarget{question-4}{%
\subsection{Question 4}\label{question-4}}

\begin{enumerate}
\def\labelenumi{\arabic{enumi})}
\item
  Import default dict and set the default value to `Not Present'. Call
  this dict\_1.
\item
  Add \texttt{int\_list}, \texttt{set\_2}, and \texttt{set\_3} to
  \texttt{dict\_1} using the object names as the key names.
\item
  Create a new dictionary, \texttt{dict\_2}, using curly bracket
  notation with \texttt{set\_1} and \texttt{az\_list} as the keys and
  values.
\item
  Invoke the default value of \texttt{dict\_1} by trying to access the
  key \texttt{az\_list}. Create a new set named \texttt{set\_4} from the
  value of
  \texttt{dict\_1{[}\textquotesingle{}az\_list\textquotesingle{}{]}}.
  What is the lenght of the difference between
  \texttt{dict\_2{[}\textquotesingle{}az\_list\textquotesingle{}{]}} and
  `set\_4'?
\item
  Update \texttt{dict\_2} with \texttt{dict\_1}. Print the value of the
  key \texttt{az\_list} from \texttt{dict\_2}. What happened?
\end{enumerate}

    \begin{tcolorbox}[breakable, size=fbox, boxrule=1pt, pad at break*=1mm,colback=cellbackground, colframe=cellborder]
\prompt{In}{incolor}{5}{\boxspacing}
\begin{Verbatim}[commandchars=\\\{\}]
\PY{c+c1}{\PYZsh{}Problem 4}
\PY{k+kn}{from} \PY{n+nn}{collections} \PY{k+kn}{import} \PY{n}{defaultdict}
\PY{k}{def} \PY{n+nf}{def\PYZus{}value}\PY{p}{(}\PY{p}{)}\PY{p}{:}
    \PY{k}{return} \PY{l+s+s2}{\PYZdq{}}\PY{l+s+s2}{Not Present}\PY{l+s+s2}{\PYZdq{}}
\PY{n}{dict\PYZus{}1} \PY{o}{=} \PY{n}{defaultdict}\PY{p}{(}\PY{n}{def\PYZus{}value}\PY{p}{)}
\PY{n}{dict\PYZus{}1}\PY{p}{[}\PY{l+s+s2}{\PYZdq{}}\PY{l+s+s2}{int\PYZus{}list}\PY{l+s+s2}{\PYZdq{}}\PY{p}{]}\PY{o}{=}\PY{n}{int\PYZus{}list}
\PY{n}{dict\PYZus{}1}\PY{p}{[}\PY{l+s+s2}{\PYZdq{}}\PY{l+s+s2}{set\PYZus{}2}\PY{l+s+s2}{\PYZdq{}}\PY{p}{]}\PY{o}{=}\PY{n}{set\PYZus{}2}
\PY{n}{dict\PYZus{}1}\PY{p}{[}\PY{l+s+s2}{\PYZdq{}}\PY{l+s+s2}{set\PYZus{}3}\PY{l+s+s2}{\PYZdq{}}\PY{p}{]}\PY{o}{=}\PY{n}{set\PYZus{}3}
\PY{n+nb}{print}\PY{p}{(}\PY{l+s+s2}{\PYZdq{}}\PY{l+s+s2}{Dictionary 1:}\PY{l+s+s2}{\PYZdq{}}\PY{p}{,}\PY{n}{dict\PYZus{}1}\PY{p}{)}
\PY{n+nb}{print}\PY{p}{(}\PY{n}{dict\PYZus{}1}\PY{p}{[}\PY{l+s+s1}{\PYZsq{}}\PY{l+s+s1}{az\PYZus{}list}\PY{l+s+s1}{\PYZsq{}}\PY{p}{]}\PY{p}{)}
\PY{n}{dict\PYZus{}2}\PY{o}{=}\PY{n}{defaultdict}\PY{p}{(}\PY{n}{def\PYZus{}value}\PY{p}{)}
\PY{n}{dict\PYZus{}2} \PY{o}{=} \PY{p}{\PYZob{}}\PY{l+s+s2}{\PYZdq{}}\PY{l+s+s2}{set\PYZus{}1}\PY{l+s+s2}{\PYZdq{}}\PY{p}{:}\PY{n}{set\PYZus{}1}\PY{p}{,}\PY{l+s+s2}{\PYZdq{}}\PY{l+s+s2}{az\PYZus{}list}\PY{l+s+s2}{\PYZdq{}}\PY{p}{:}\PY{n}{az\PYZus{}list}\PY{p}{\PYZcb{}}
\PY{n+nb}{print}\PY{p}{(}\PY{l+s+s2}{\PYZdq{}}\PY{l+s+s2}{Dictionary 2:}\PY{l+s+s2}{\PYZdq{}}\PY{p}{,}\PY{n}{dict\PYZus{}2}\PY{p}{)}
\PY{n}{set\PYZus{}4}\PY{o}{=}\PY{n+nb}{set}\PY{p}{(}\PY{n}{dict\PYZus{}1}\PY{p}{[}\PY{l+s+s1}{\PYZsq{}}\PY{l+s+s1}{az\PYZus{}list}\PY{l+s+s1}{\PYZsq{}}\PY{p}{]}\PY{p}{)}
\PY{n+nb}{print}\PY{p}{(}\PY{l+s+s2}{\PYZdq{}}\PY{l+s+s2}{Set 4:}\PY{l+s+s2}{\PYZdq{}}\PY{p}{,} \PY{n}{set\PYZus{}4}\PY{p}{)}
\PY{n+nb}{print}\PY{p}{(}\PY{l+s+s2}{\PYZdq{}}\PY{l+s+s2}{Since }\PY{l+s+s2}{\PYZsq{}}\PY{l+s+s2}{az\PYZus{}list}\PY{l+s+s2}{\PYZsq{}}\PY{l+s+s2}{ is not a key in dict\PYZus{}1, this just took the default value }\PY{l+s+s2}{\PYZsq{}}\PY{l+s+s2}{Not Present}\PY{l+s+s2}{\PYZsq{}}\PY{l+s+s2}{ and turned it into a set by letter!}\PY{l+s+s2}{\PYZdq{}}\PY{p}{)}
\PY{n}{set\PYZus{}diff}\PY{o}{=}\PY{n+nb}{set}\PY{p}{(}\PY{n}{dict\PYZus{}2}\PY{p}{[}\PY{l+s+s1}{\PYZsq{}}\PY{l+s+s1}{az\PYZus{}list}\PY{l+s+s1}{\PYZsq{}}\PY{p}{]}\PY{p}{)}\PY{o}{.}\PY{n}{symmetric\PYZus{}difference}\PY{p}{(}\PY{n}{set\PYZus{}4}\PY{p}{)}
\PY{n+nb}{print}\PY{p}{(}\PY{l+s+s2}{\PYZdq{}}\PY{l+s+s2}{The length of the difference between dict\PYZus{}2[}\PY{l+s+s2}{\PYZsq{}}\PY{l+s+s2}{az\PYZus{}list}\PY{l+s+s2}{\PYZsq{}}\PY{l+s+s2}{] and dict\PYZus{}2 is}\PY{l+s+s2}{\PYZdq{}}\PY{p}{,} \PY{n+nb}{len}\PY{p}{(}\PY{n}{set\PYZus{}diff}\PY{p}{)}\PY{p}{,}\PY{l+s+s2}{\PYZdq{}}\PY{l+s+s2}{elements}\PY{l+s+s2}{\PYZdq{}}\PY{p}{)}
\PY{n}{dict\PYZus{}2}\PY{o}{.}\PY{n}{update}\PY{p}{(}\PY{n}{dict\PYZus{}1}\PY{p}{)}
\PY{n+nb}{print}\PY{p}{(}\PY{n}{dict\PYZus{}2}\PY{p}{[}\PY{l+s+s1}{\PYZsq{}}\PY{l+s+s1}{az\PYZus{}list}\PY{l+s+s1}{\PYZsq{}}\PY{p}{]}\PY{p}{)}
\PY{n+nb}{print}\PY{p}{(}\PY{l+s+s2}{\PYZdq{}}\PY{l+s+s2}{It appears that when the dictionary was updated with the contents of dict\PYZus{}1, the value for key }\PY{l+s+s2}{\PYZsq{}}\PY{l+s+s2}{az\PYZus{}list}\PY{l+s+s2}{\PYZsq{}}\PY{l+s+s2}{,}\PY{l+s+s2}{\PYZdq{}}\PY{p}{)}
\PY{n+nb}{print}\PY{p}{(}\PY{l+s+s2}{\PYZdq{}}\PY{l+s+s2}{which used to be our list of capital letters, was replaced with }\PY{l+s+s2}{\PYZsq{}}\PY{l+s+s2}{Not Present}\PY{l+s+s2}{\PYZsq{}}\PY{l+s+s2}{ which is the default value.}\PY{l+s+s2}{\PYZdq{}}\PY{p}{)}
\PY{n+nb}{print}\PY{p}{(}\PY{l+s+s2}{\PYZdq{}}\PY{l+s+s2}{Even after researching, it is unclear to me why!}\PY{l+s+s2}{\PYZdq{}}\PY{p}{)}
\end{Verbatim}
\end{tcolorbox}

    \begin{Verbatim}[commandchars=\\\{\}]
Dictionary 1: defaultdict(<function def\_value at 0x7fd758a2c820>, \{'int\_list':
[1, 13, 15, 0, 6, 14, 15, 8, 5, 1, 15, 10, 2, 7, 11, 1, 13, 4, 11, 12, 13, 9, 8,
14, 5, 9, 11, 4, 14, 7, 14, 12, 1, 0, 7, 4, 6, 9, 11, 7, 10, 14, 13, 15, 2, 10,
5, 7, 13, 7], 'set\_2': \{0, 1, 2, 4, 5, 6, 7, 8, 9, 10, 11, 12, 13, 14, 15\},
'set\_3': \{0, 3, 6, 7, 8, 9, 10, 11, 12, 13, 14, 15\}\})
Not Present
Dictionary 2: \{'set\_1': \{1, 2, 3, 4, 5\}, 'az\_list': ['A', 'B', 'C', 'D', 'E',
'F', 'G', 'H', 'I', 'J', 'K', 'L', 'M', 'N', 'O', 'P', 'Q', 'R', 'S', 'T', 'U',
'V', 'W', 'X', 'Y', 'Z']\}
Set 4: \{'t', 'r', ' ', 'n', 'o', 'e', 's', 'P', 'N'\}
Since 'az\_list' is not a key in dict\_1, this just took the default value 'Not
Present' and turned it into a set by letter!
The length of the difference between dict\_2['az\_list'] and dict\_2 is 31 elements
Not Present
It appears that when the dictionary was updated with the contents of dict\_1, the
value for key 'az\_list',
which used to be our list of capital letters, was replaced with 'Not Present'
which is the default value.
Even after researching, it is unclear to me why!
    \end{Verbatim}

    \begin{tcolorbox}[breakable, size=fbox, boxrule=1pt, pad at break*=1mm,colback=cellbackground, colframe=cellborder]
\prompt{In}{incolor}{ }{\boxspacing}
\begin{Verbatim}[commandchars=\\\{\}]

\end{Verbatim}
\end{tcolorbox}


    % Add a bibliography block to the postdoc
    
    
    
\end{document}
