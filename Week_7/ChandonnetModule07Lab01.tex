\documentclass[11pt]{article}

    \usepackage[breakable]{tcolorbox}
    \usepackage{parskip} % Stop auto-indenting (to mimic markdown behaviour)
    

    % Basic figure setup, for now with no caption control since it's done
    % automatically by Pandoc (which extracts ![](path) syntax from Markdown).
    \usepackage{graphicx}
    % Maintain compatibility with old templates. Remove in nbconvert 6.0
    \let\Oldincludegraphics\includegraphics
    % Ensure that by default, figures have no caption (until we provide a
    % proper Figure object with a Caption API and a way to capture that
    % in the conversion process - todo).
    \usepackage{caption}
    \DeclareCaptionFormat{nocaption}{}
    \captionsetup{format=nocaption,aboveskip=0pt,belowskip=0pt}

    \usepackage{float}
    \floatplacement{figure}{H} % forces figures to be placed at the correct location
    \usepackage{xcolor} % Allow colors to be defined
    \usepackage{enumerate} % Needed for markdown enumerations to work
    \usepackage{geometry} % Used to adjust the document margins
    \usepackage{amsmath} % Equations
    \usepackage{amssymb} % Equations
    \usepackage{textcomp} % defines textquotesingle
    % Hack from http://tex.stackexchange.com/a/47451/13684:
    \AtBeginDocument{%
        \def\PYZsq{\textquotesingle}% Upright quotes in Pygmentized code
    }
    \usepackage{upquote} % Upright quotes for verbatim code
    \usepackage{eurosym} % defines \euro

    \usepackage{iftex}
    \ifPDFTeX
        \usepackage[T1]{fontenc}
        \IfFileExists{alphabeta.sty}{
              \usepackage{alphabeta}
          }{
              \usepackage[mathletters]{ucs}
              \usepackage[utf8x]{inputenc}
          }
    \else
        \usepackage{fontspec}
        \usepackage{unicode-math}
    \fi

    \usepackage{fancyvrb} % verbatim replacement that allows latex
    \usepackage{grffile} % extends the file name processing of package graphics
                         % to support a larger range
    \makeatletter % fix for old versions of grffile with XeLaTeX
    \@ifpackagelater{grffile}{2019/11/01}
    {
      % Do nothing on new versions
    }
    {
      \def\Gread@@xetex#1{%
        \IfFileExists{"\Gin@base".bb}%
        {\Gread@eps{\Gin@base.bb}}%
        {\Gread@@xetex@aux#1}%
      }
    }
    \makeatother
    \usepackage[Export]{adjustbox} % Used to constrain images to a maximum size
    \adjustboxset{max size={0.9\linewidth}{0.9\paperheight}}

    % The hyperref package gives us a pdf with properly built
    % internal navigation ('pdf bookmarks' for the table of contents,
    % internal cross-reference links, web links for URLs, etc.)
    \usepackage{hyperref}
    % The default LaTeX title has an obnoxious amount of whitespace. By default,
    % titling removes some of it. It also provides customization options.
    \usepackage{titling}
    \usepackage{longtable} % longtable support required by pandoc >1.10
    \usepackage{booktabs}  % table support for pandoc > 1.12.2
    \usepackage{array}     % table support for pandoc >= 2.11.3
    \usepackage{calc}      % table minipage width calculation for pandoc >= 2.11.1
    \usepackage[inline]{enumitem} % IRkernel/repr support (it uses the enumerate* environment)
    \usepackage[normalem]{ulem} % ulem is needed to support strikethroughs (\sout)
                                % normalem makes italics be italics, not underlines
    \usepackage{mathrsfs}
    

    
    % Colors for the hyperref package
    \definecolor{urlcolor}{rgb}{0,.145,.698}
    \definecolor{linkcolor}{rgb}{.71,0.21,0.01}
    \definecolor{citecolor}{rgb}{.12,.54,.11}

    % ANSI colors
    \definecolor{ansi-black}{HTML}{3E424D}
    \definecolor{ansi-black-intense}{HTML}{282C36}
    \definecolor{ansi-red}{HTML}{E75C58}
    \definecolor{ansi-red-intense}{HTML}{B22B31}
    \definecolor{ansi-green}{HTML}{00A250}
    \definecolor{ansi-green-intense}{HTML}{007427}
    \definecolor{ansi-yellow}{HTML}{DDB62B}
    \definecolor{ansi-yellow-intense}{HTML}{B27D12}
    \definecolor{ansi-blue}{HTML}{208FFB}
    \definecolor{ansi-blue-intense}{HTML}{0065CA}
    \definecolor{ansi-magenta}{HTML}{D160C4}
    \definecolor{ansi-magenta-intense}{HTML}{A03196}
    \definecolor{ansi-cyan}{HTML}{60C6C8}
    \definecolor{ansi-cyan-intense}{HTML}{258F8F}
    \definecolor{ansi-white}{HTML}{C5C1B4}
    \definecolor{ansi-white-intense}{HTML}{A1A6B2}
    \definecolor{ansi-default-inverse-fg}{HTML}{FFFFFF}
    \definecolor{ansi-default-inverse-bg}{HTML}{000000}

    % common color for the border for error outputs.
    \definecolor{outerrorbackground}{HTML}{FFDFDF}

    % commands and environments needed by pandoc snippets
    % extracted from the output of `pandoc -s`
    \providecommand{\tightlist}{%
      \setlength{\itemsep}{0pt}\setlength{\parskip}{0pt}}
    \DefineVerbatimEnvironment{Highlighting}{Verbatim}{commandchars=\\\{\}}
    % Add ',fontsize=\small' for more characters per line
    \newenvironment{Shaded}{}{}
    \newcommand{\KeywordTok}[1]{\textcolor[rgb]{0.00,0.44,0.13}{\textbf{{#1}}}}
    \newcommand{\DataTypeTok}[1]{\textcolor[rgb]{0.56,0.13,0.00}{{#1}}}
    \newcommand{\DecValTok}[1]{\textcolor[rgb]{0.25,0.63,0.44}{{#1}}}
    \newcommand{\BaseNTok}[1]{\textcolor[rgb]{0.25,0.63,0.44}{{#1}}}
    \newcommand{\FloatTok}[1]{\textcolor[rgb]{0.25,0.63,0.44}{{#1}}}
    \newcommand{\CharTok}[1]{\textcolor[rgb]{0.25,0.44,0.63}{{#1}}}
    \newcommand{\StringTok}[1]{\textcolor[rgb]{0.25,0.44,0.63}{{#1}}}
    \newcommand{\CommentTok}[1]{\textcolor[rgb]{0.38,0.63,0.69}{\textit{{#1}}}}
    \newcommand{\OtherTok}[1]{\textcolor[rgb]{0.00,0.44,0.13}{{#1}}}
    \newcommand{\AlertTok}[1]{\textcolor[rgb]{1.00,0.00,0.00}{\textbf{{#1}}}}
    \newcommand{\FunctionTok}[1]{\textcolor[rgb]{0.02,0.16,0.49}{{#1}}}
    \newcommand{\RegionMarkerTok}[1]{{#1}}
    \newcommand{\ErrorTok}[1]{\textcolor[rgb]{1.00,0.00,0.00}{\textbf{{#1}}}}
    \newcommand{\NormalTok}[1]{{#1}}

    % Additional commands for more recent versions of Pandoc
    \newcommand{\ConstantTok}[1]{\textcolor[rgb]{0.53,0.00,0.00}{{#1}}}
    \newcommand{\SpecialCharTok}[1]{\textcolor[rgb]{0.25,0.44,0.63}{{#1}}}
    \newcommand{\VerbatimStringTok}[1]{\textcolor[rgb]{0.25,0.44,0.63}{{#1}}}
    \newcommand{\SpecialStringTok}[1]{\textcolor[rgb]{0.73,0.40,0.53}{{#1}}}
    \newcommand{\ImportTok}[1]{{#1}}
    \newcommand{\DocumentationTok}[1]{\textcolor[rgb]{0.73,0.13,0.13}{\textit{{#1}}}}
    \newcommand{\AnnotationTok}[1]{\textcolor[rgb]{0.38,0.63,0.69}{\textbf{\textit{{#1}}}}}
    \newcommand{\CommentVarTok}[1]{\textcolor[rgb]{0.38,0.63,0.69}{\textbf{\textit{{#1}}}}}
    \newcommand{\VariableTok}[1]{\textcolor[rgb]{0.10,0.09,0.49}{{#1}}}
    \newcommand{\ControlFlowTok}[1]{\textcolor[rgb]{0.00,0.44,0.13}{\textbf{{#1}}}}
    \newcommand{\OperatorTok}[1]{\textcolor[rgb]{0.40,0.40,0.40}{{#1}}}
    \newcommand{\BuiltInTok}[1]{{#1}}
    \newcommand{\ExtensionTok}[1]{{#1}}
    \newcommand{\PreprocessorTok}[1]{\textcolor[rgb]{0.74,0.48,0.00}{{#1}}}
    \newcommand{\AttributeTok}[1]{\textcolor[rgb]{0.49,0.56,0.16}{{#1}}}
    \newcommand{\InformationTok}[1]{\textcolor[rgb]{0.38,0.63,0.69}{\textbf{\textit{{#1}}}}}
    \newcommand{\WarningTok}[1]{\textcolor[rgb]{0.38,0.63,0.69}{\textbf{\textit{{#1}}}}}


    % Define a nice break command that doesn't care if a line doesn't already
    % exist.
    \def\br{\hspace*{\fill} \\* }
    % Math Jax compatibility definitions
    \def\gt{>}
    \def\lt{<}
    \let\Oldtex\TeX
    \let\Oldlatex\LaTeX
    \renewcommand{\TeX}{\textrm{\Oldtex}}
    \renewcommand{\LaTeX}{\textrm{\Oldlatex}}
    % Document parameters
    % Document title
    \title{ChandonnetModule07Lab01}
    
    
    
    
    
% Pygments definitions
\makeatletter
\def\PY@reset{\let\PY@it=\relax \let\PY@bf=\relax%
    \let\PY@ul=\relax \let\PY@tc=\relax%
    \let\PY@bc=\relax \let\PY@ff=\relax}
\def\PY@tok#1{\csname PY@tok@#1\endcsname}
\def\PY@toks#1+{\ifx\relax#1\empty\else%
    \PY@tok{#1}\expandafter\PY@toks\fi}
\def\PY@do#1{\PY@bc{\PY@tc{\PY@ul{%
    \PY@it{\PY@bf{\PY@ff{#1}}}}}}}
\def\PY#1#2{\PY@reset\PY@toks#1+\relax+\PY@do{#2}}

\@namedef{PY@tok@w}{\def\PY@tc##1{\textcolor[rgb]{0.73,0.73,0.73}{##1}}}
\@namedef{PY@tok@c}{\let\PY@it=\textit\def\PY@tc##1{\textcolor[rgb]{0.24,0.48,0.48}{##1}}}
\@namedef{PY@tok@cp}{\def\PY@tc##1{\textcolor[rgb]{0.61,0.40,0.00}{##1}}}
\@namedef{PY@tok@k}{\let\PY@bf=\textbf\def\PY@tc##1{\textcolor[rgb]{0.00,0.50,0.00}{##1}}}
\@namedef{PY@tok@kp}{\def\PY@tc##1{\textcolor[rgb]{0.00,0.50,0.00}{##1}}}
\@namedef{PY@tok@kt}{\def\PY@tc##1{\textcolor[rgb]{0.69,0.00,0.25}{##1}}}
\@namedef{PY@tok@o}{\def\PY@tc##1{\textcolor[rgb]{0.40,0.40,0.40}{##1}}}
\@namedef{PY@tok@ow}{\let\PY@bf=\textbf\def\PY@tc##1{\textcolor[rgb]{0.67,0.13,1.00}{##1}}}
\@namedef{PY@tok@nb}{\def\PY@tc##1{\textcolor[rgb]{0.00,0.50,0.00}{##1}}}
\@namedef{PY@tok@nf}{\def\PY@tc##1{\textcolor[rgb]{0.00,0.00,1.00}{##1}}}
\@namedef{PY@tok@nc}{\let\PY@bf=\textbf\def\PY@tc##1{\textcolor[rgb]{0.00,0.00,1.00}{##1}}}
\@namedef{PY@tok@nn}{\let\PY@bf=\textbf\def\PY@tc##1{\textcolor[rgb]{0.00,0.00,1.00}{##1}}}
\@namedef{PY@tok@ne}{\let\PY@bf=\textbf\def\PY@tc##1{\textcolor[rgb]{0.80,0.25,0.22}{##1}}}
\@namedef{PY@tok@nv}{\def\PY@tc##1{\textcolor[rgb]{0.10,0.09,0.49}{##1}}}
\@namedef{PY@tok@no}{\def\PY@tc##1{\textcolor[rgb]{0.53,0.00,0.00}{##1}}}
\@namedef{PY@tok@nl}{\def\PY@tc##1{\textcolor[rgb]{0.46,0.46,0.00}{##1}}}
\@namedef{PY@tok@ni}{\let\PY@bf=\textbf\def\PY@tc##1{\textcolor[rgb]{0.44,0.44,0.44}{##1}}}
\@namedef{PY@tok@na}{\def\PY@tc##1{\textcolor[rgb]{0.41,0.47,0.13}{##1}}}
\@namedef{PY@tok@nt}{\let\PY@bf=\textbf\def\PY@tc##1{\textcolor[rgb]{0.00,0.50,0.00}{##1}}}
\@namedef{PY@tok@nd}{\def\PY@tc##1{\textcolor[rgb]{0.67,0.13,1.00}{##1}}}
\@namedef{PY@tok@s}{\def\PY@tc##1{\textcolor[rgb]{0.73,0.13,0.13}{##1}}}
\@namedef{PY@tok@sd}{\let\PY@it=\textit\def\PY@tc##1{\textcolor[rgb]{0.73,0.13,0.13}{##1}}}
\@namedef{PY@tok@si}{\let\PY@bf=\textbf\def\PY@tc##1{\textcolor[rgb]{0.64,0.35,0.47}{##1}}}
\@namedef{PY@tok@se}{\let\PY@bf=\textbf\def\PY@tc##1{\textcolor[rgb]{0.67,0.36,0.12}{##1}}}
\@namedef{PY@tok@sr}{\def\PY@tc##1{\textcolor[rgb]{0.64,0.35,0.47}{##1}}}
\@namedef{PY@tok@ss}{\def\PY@tc##1{\textcolor[rgb]{0.10,0.09,0.49}{##1}}}
\@namedef{PY@tok@sx}{\def\PY@tc##1{\textcolor[rgb]{0.00,0.50,0.00}{##1}}}
\@namedef{PY@tok@m}{\def\PY@tc##1{\textcolor[rgb]{0.40,0.40,0.40}{##1}}}
\@namedef{PY@tok@gh}{\let\PY@bf=\textbf\def\PY@tc##1{\textcolor[rgb]{0.00,0.00,0.50}{##1}}}
\@namedef{PY@tok@gu}{\let\PY@bf=\textbf\def\PY@tc##1{\textcolor[rgb]{0.50,0.00,0.50}{##1}}}
\@namedef{PY@tok@gd}{\def\PY@tc##1{\textcolor[rgb]{0.63,0.00,0.00}{##1}}}
\@namedef{PY@tok@gi}{\def\PY@tc##1{\textcolor[rgb]{0.00,0.52,0.00}{##1}}}
\@namedef{PY@tok@gr}{\def\PY@tc##1{\textcolor[rgb]{0.89,0.00,0.00}{##1}}}
\@namedef{PY@tok@ge}{\let\PY@it=\textit}
\@namedef{PY@tok@gs}{\let\PY@bf=\textbf}
\@namedef{PY@tok@gp}{\let\PY@bf=\textbf\def\PY@tc##1{\textcolor[rgb]{0.00,0.00,0.50}{##1}}}
\@namedef{PY@tok@go}{\def\PY@tc##1{\textcolor[rgb]{0.44,0.44,0.44}{##1}}}
\@namedef{PY@tok@gt}{\def\PY@tc##1{\textcolor[rgb]{0.00,0.27,0.87}{##1}}}
\@namedef{PY@tok@err}{\def\PY@bc##1{{\setlength{\fboxsep}{\string -\fboxrule}\fcolorbox[rgb]{1.00,0.00,0.00}{1,1,1}{\strut ##1}}}}
\@namedef{PY@tok@kc}{\let\PY@bf=\textbf\def\PY@tc##1{\textcolor[rgb]{0.00,0.50,0.00}{##1}}}
\@namedef{PY@tok@kd}{\let\PY@bf=\textbf\def\PY@tc##1{\textcolor[rgb]{0.00,0.50,0.00}{##1}}}
\@namedef{PY@tok@kn}{\let\PY@bf=\textbf\def\PY@tc##1{\textcolor[rgb]{0.00,0.50,0.00}{##1}}}
\@namedef{PY@tok@kr}{\let\PY@bf=\textbf\def\PY@tc##1{\textcolor[rgb]{0.00,0.50,0.00}{##1}}}
\@namedef{PY@tok@bp}{\def\PY@tc##1{\textcolor[rgb]{0.00,0.50,0.00}{##1}}}
\@namedef{PY@tok@fm}{\def\PY@tc##1{\textcolor[rgb]{0.00,0.00,1.00}{##1}}}
\@namedef{PY@tok@vc}{\def\PY@tc##1{\textcolor[rgb]{0.10,0.09,0.49}{##1}}}
\@namedef{PY@tok@vg}{\def\PY@tc##1{\textcolor[rgb]{0.10,0.09,0.49}{##1}}}
\@namedef{PY@tok@vi}{\def\PY@tc##1{\textcolor[rgb]{0.10,0.09,0.49}{##1}}}
\@namedef{PY@tok@vm}{\def\PY@tc##1{\textcolor[rgb]{0.10,0.09,0.49}{##1}}}
\@namedef{PY@tok@sa}{\def\PY@tc##1{\textcolor[rgb]{0.73,0.13,0.13}{##1}}}
\@namedef{PY@tok@sb}{\def\PY@tc##1{\textcolor[rgb]{0.73,0.13,0.13}{##1}}}
\@namedef{PY@tok@sc}{\def\PY@tc##1{\textcolor[rgb]{0.73,0.13,0.13}{##1}}}
\@namedef{PY@tok@dl}{\def\PY@tc##1{\textcolor[rgb]{0.73,0.13,0.13}{##1}}}
\@namedef{PY@tok@s2}{\def\PY@tc##1{\textcolor[rgb]{0.73,0.13,0.13}{##1}}}
\@namedef{PY@tok@sh}{\def\PY@tc##1{\textcolor[rgb]{0.73,0.13,0.13}{##1}}}
\@namedef{PY@tok@s1}{\def\PY@tc##1{\textcolor[rgb]{0.73,0.13,0.13}{##1}}}
\@namedef{PY@tok@mb}{\def\PY@tc##1{\textcolor[rgb]{0.40,0.40,0.40}{##1}}}
\@namedef{PY@tok@mf}{\def\PY@tc##1{\textcolor[rgb]{0.40,0.40,0.40}{##1}}}
\@namedef{PY@tok@mh}{\def\PY@tc##1{\textcolor[rgb]{0.40,0.40,0.40}{##1}}}
\@namedef{PY@tok@mi}{\def\PY@tc##1{\textcolor[rgb]{0.40,0.40,0.40}{##1}}}
\@namedef{PY@tok@il}{\def\PY@tc##1{\textcolor[rgb]{0.40,0.40,0.40}{##1}}}
\@namedef{PY@tok@mo}{\def\PY@tc##1{\textcolor[rgb]{0.40,0.40,0.40}{##1}}}
\@namedef{PY@tok@ch}{\let\PY@it=\textit\def\PY@tc##1{\textcolor[rgb]{0.24,0.48,0.48}{##1}}}
\@namedef{PY@tok@cm}{\let\PY@it=\textit\def\PY@tc##1{\textcolor[rgb]{0.24,0.48,0.48}{##1}}}
\@namedef{PY@tok@cpf}{\let\PY@it=\textit\def\PY@tc##1{\textcolor[rgb]{0.24,0.48,0.48}{##1}}}
\@namedef{PY@tok@c1}{\let\PY@it=\textit\def\PY@tc##1{\textcolor[rgb]{0.24,0.48,0.48}{##1}}}
\@namedef{PY@tok@cs}{\let\PY@it=\textit\def\PY@tc##1{\textcolor[rgb]{0.24,0.48,0.48}{##1}}}

\def\PYZbs{\char`\\}
\def\PYZus{\char`\_}
\def\PYZob{\char`\{}
\def\PYZcb{\char`\}}
\def\PYZca{\char`\^}
\def\PYZam{\char`\&}
\def\PYZlt{\char`\<}
\def\PYZgt{\char`\>}
\def\PYZsh{\char`\#}
\def\PYZpc{\char`\%}
\def\PYZdl{\char`\$}
\def\PYZhy{\char`\-}
\def\PYZsq{\char`\'}
\def\PYZdq{\char`\"}
\def\PYZti{\char`\~}
% for compatibility with earlier versions
\def\PYZat{@}
\def\PYZlb{[}
\def\PYZrb{]}
\makeatother


    % For linebreaks inside Verbatim environment from package fancyvrb.
    \makeatletter
        \newbox\Wrappedcontinuationbox
        \newbox\Wrappedvisiblespacebox
        \newcommand*\Wrappedvisiblespace {\textcolor{red}{\textvisiblespace}}
        \newcommand*\Wrappedcontinuationsymbol {\textcolor{red}{\llap{\tiny$\m@th\hookrightarrow$}}}
        \newcommand*\Wrappedcontinuationindent {3ex }
        \newcommand*\Wrappedafterbreak {\kern\Wrappedcontinuationindent\copy\Wrappedcontinuationbox}
        % Take advantage of the already applied Pygments mark-up to insert
        % potential linebreaks for TeX processing.
        %        {, <, #, %, $, ' and ": go to next line.
        %        _, }, ^, &, >, - and ~: stay at end of broken line.
        % Use of \textquotesingle for straight quote.
        \newcommand*\Wrappedbreaksatspecials {%
            \def\PYGZus{\discretionary{\char`\_}{\Wrappedafterbreak}{\char`\_}}%
            \def\PYGZob{\discretionary{}{\Wrappedafterbreak\char`\{}{\char`\{}}%
            \def\PYGZcb{\discretionary{\char`\}}{\Wrappedafterbreak}{\char`\}}}%
            \def\PYGZca{\discretionary{\char`\^}{\Wrappedafterbreak}{\char`\^}}%
            \def\PYGZam{\discretionary{\char`\&}{\Wrappedafterbreak}{\char`\&}}%
            \def\PYGZlt{\discretionary{}{\Wrappedafterbreak\char`\<}{\char`\<}}%
            \def\PYGZgt{\discretionary{\char`\>}{\Wrappedafterbreak}{\char`\>}}%
            \def\PYGZsh{\discretionary{}{\Wrappedafterbreak\char`\#}{\char`\#}}%
            \def\PYGZpc{\discretionary{}{\Wrappedafterbreak\char`\%}{\char`\%}}%
            \def\PYGZdl{\discretionary{}{\Wrappedafterbreak\char`\$}{\char`\$}}%
            \def\PYGZhy{\discretionary{\char`\-}{\Wrappedafterbreak}{\char`\-}}%
            \def\PYGZsq{\discretionary{}{\Wrappedafterbreak\textquotesingle}{\textquotesingle}}%
            \def\PYGZdq{\discretionary{}{\Wrappedafterbreak\char`\"}{\char`\"}}%
            \def\PYGZti{\discretionary{\char`\~}{\Wrappedafterbreak}{\char`\~}}%
        }
        % Some characters . , ; ? ! / are not pygmentized.
        % This macro makes them "active" and they will insert potential linebreaks
        \newcommand*\Wrappedbreaksatpunct {%
            \lccode`\~`\.\lowercase{\def~}{\discretionary{\hbox{\char`\.}}{\Wrappedafterbreak}{\hbox{\char`\.}}}%
            \lccode`\~`\,\lowercase{\def~}{\discretionary{\hbox{\char`\,}}{\Wrappedafterbreak}{\hbox{\char`\,}}}%
            \lccode`\~`\;\lowercase{\def~}{\discretionary{\hbox{\char`\;}}{\Wrappedafterbreak}{\hbox{\char`\;}}}%
            \lccode`\~`\:\lowercase{\def~}{\discretionary{\hbox{\char`\:}}{\Wrappedafterbreak}{\hbox{\char`\:}}}%
            \lccode`\~`\?\lowercase{\def~}{\discretionary{\hbox{\char`\?}}{\Wrappedafterbreak}{\hbox{\char`\?}}}%
            \lccode`\~`\!\lowercase{\def~}{\discretionary{\hbox{\char`\!}}{\Wrappedafterbreak}{\hbox{\char`\!}}}%
            \lccode`\~`\/\lowercase{\def~}{\discretionary{\hbox{\char`\/}}{\Wrappedafterbreak}{\hbox{\char`\/}}}%
            \catcode`\.\active
            \catcode`\,\active
            \catcode`\;\active
            \catcode`\:\active
            \catcode`\?\active
            \catcode`\!\active
            \catcode`\/\active
            \lccode`\~`\~
        }
    \makeatother

    \let\OriginalVerbatim=\Verbatim
    \makeatletter
    \renewcommand{\Verbatim}[1][1]{%
        %\parskip\z@skip
        \sbox\Wrappedcontinuationbox {\Wrappedcontinuationsymbol}%
        \sbox\Wrappedvisiblespacebox {\FV@SetupFont\Wrappedvisiblespace}%
        \def\FancyVerbFormatLine ##1{\hsize\linewidth
            \vtop{\raggedright\hyphenpenalty\z@\exhyphenpenalty\z@
                \doublehyphendemerits\z@\finalhyphendemerits\z@
                \strut ##1\strut}%
        }%
        % If the linebreak is at a space, the latter will be displayed as visible
        % space at end of first line, and a continuation symbol starts next line.
        % Stretch/shrink are however usually zero for typewriter font.
        \def\FV@Space {%
            \nobreak\hskip\z@ plus\fontdimen3\font minus\fontdimen4\font
            \discretionary{\copy\Wrappedvisiblespacebox}{\Wrappedafterbreak}
            {\kern\fontdimen2\font}%
        }%

        % Allow breaks at special characters using \PYG... macros.
        \Wrappedbreaksatspecials
        % Breaks at punctuation characters . , ; ? ! and / need catcode=\active
        \OriginalVerbatim[#1,codes*=\Wrappedbreaksatpunct]%
    }
    \makeatother

    % Exact colors from NB
    \definecolor{incolor}{HTML}{303F9F}
    \definecolor{outcolor}{HTML}{D84315}
    \definecolor{cellborder}{HTML}{CFCFCF}
    \definecolor{cellbackground}{HTML}{F7F7F7}

    % prompt
    \makeatletter
    \newcommand{\boxspacing}{\kern\kvtcb@left@rule\kern\kvtcb@boxsep}
    \makeatother
    \newcommand{\prompt}[4]{
        {\ttfamily\llap{{\color{#2}[#3]:\hspace{3pt}#4}}\vspace{-\baselineskip}}
    }
    

    
    % Prevent overflowing lines due to hard-to-break entities
    \sloppy
    % Setup hyperref package
    \hypersetup{
      breaklinks=true,  % so long urls are correctly broken across lines
      colorlinks=true,
      urlcolor=urlcolor,
      linkcolor=linkcolor,
      citecolor=citecolor,
      }
    % Slightly bigger margins than the latex defaults
    
    \geometry{verbose,tmargin=1in,bmargin=1in,lmargin=1in,rmargin=1in}
    
    

\begin{document}
    
    \maketitle
    
    

    
    \hypertarget{assignment-7}{%
\section{Assignment 7}\label{assignment-7}}

\hypertarget{ray-chandonnet}{%
\subsubsection{Ray Chandonnet}\label{ray-chandonnet}}

\hypertarget{december-8-2022}{%
\subsubsection{December 8, 2022}\label{december-8-2022}}

    \hypertarget{question-1}{%
\subsection{Question 1}\label{question-1}}

A palindrome is a word, phrase, or sequence that is the same spelled
forward as it is backwards. Write a function using a for-loop to
determine if a string is a palindrome. Your function should only have
one argument.

    \begin{tcolorbox}[breakable, size=fbox, boxrule=1pt, pad at break*=1mm,colback=cellbackground, colframe=cellborder]
\prompt{In}{incolor}{10}{\boxspacing}
\begin{Verbatim}[commandchars=\\\{\}]
\PY{c+c1}{\PYZsh{} My basic logic is:}
\PY{c+c1}{\PYZsh{}   \PYZhy{} I set a palindrome Boolean to True; }
\PY{c+c1}{\PYZsh{}   \PYZhy{} I determine the midpoint of the string (truncated as an integer for strings of odd length}
\PY{c+c1}{\PYZsh{}   \PYZhy{} I use one counter in a for loop that iterates through the first half of the string}
\PY{c+c1}{\PYZsh{}   \PYZhy{} I use a secound counter that starts at the end of the string and is decremented for each loop run}
\PY{c+c1}{\PYZsh{}   \PYZhy{} In each loop run I compare the character at position of counter 1 to character at position counter 2}
\PY{c+c1}{\PYZsh{}     so for example, if string is 7 characters long, I compare characters 1 and 7, 2 and 6, \PYZam{} 3 and 5; }
\PY{c+c1}{\PYZsh{}     if string is 8 characters long, I compare characters 1 and 8, 2 and 7, 3 and 6, \PYZam{} 4 and 5; }
\PY{c+c1}{\PYZsh{}   \PYZhy{} As soon as I encounter a case where the compared characters don\PYZsq{}t match, I know it\PYZsq{}s not a palindrome, }
\PY{c+c1}{\PYZsh{}     so I set the palindrome Boolean to False and \PYZdq{}break\PYZdq{} the for loop}
\PY{c+c1}{\PYZsh{}   \PYZhy{} If I get all the way through the for loop without breaking, it is a palindrome and the palindrome}
\PY{c+c1}{\PYZsh{}     Boolean has remained TRUE}
\PY{c+c1}{\PYZsh{}}
\PY{c+c1}{\PYZsh{} A couple of comments:}
\PY{c+c1}{\PYZsh{}}
\PY{c+c1}{\PYZsh{} 1) The reason I only iterate through half the string is that it saves work \PYZhy{} comparing characters 7 and 1}
\PY{c+c1}{\PYZsh{}    is the same as comparing characters 1 and 7 so I\PYZsq{}d be repeating evaluations already done if I looped}
\PY{c+c1}{\PYZsh{}    through every character}
\PY{c+c1}{\PYZsh{}}
\PY{c+c1}{\PYZsh{} 2) I personally HATE \PYZdq{}breaking\PYZdq{} a for loop as a coding construct \PYZhy{} I did it in order to save work (no need to }
\PY{c+c1}{\PYZsh{}    continue comparing characters once we have found a character mismatch) but I feel that is bad use of a For }
\PY{c+c1}{\PYZsh{}    loop;  A better construct I think would be to use a while loop, as in Question 2}
\PY{c+c1}{\PYZsh{}}
\PY{k}{def} \PY{n+nf}{palindrome\PYZus{}for}\PY{p}{(}\PY{n}{eval\PYZus{}str}\PY{p}{)}\PY{p}{:}
    \PY{c+c1}{\PYZsh{}}
    \PY{c+c1}{\PYZsh{} This function determines whether a string is a palindrome}
    \PY{c+c1}{\PYZsh{} Input:  eval\PYZus{}str = The string to be evaluated}
    \PY{c+c1}{\PYZsh{} Output:  The function returns a Boolean value True if eval\PYZus{}str is a palindrome, False if not}
    \PY{c+c1}{\PYZsh{}}
    \PY{k}{if} \PY{n+nb}{len}\PY{p}{(}\PY{n}{eval\PYZus{}str}\PY{p}{)}\PY{o}{==}\PY{l+m+mi}{0}\PY{p}{:}  \PY{c+c1}{\PYZsh{} check for null string}
        \PY{n}{is\PYZus{}palindrome}\PY{o}{=}\PY{k+kc}{False}   \PY{c+c1}{\PYZsh{}return False if no string provided; Though I supposed you could argue null string}
                              \PY{c+c1}{\PYZsh{} IS in fact a palindrome :)}
    \PY{k}{else}\PY{p}{:}        
        \PY{n}{is\PYZus{}palindrome}\PY{o}{=}\PY{k+kc}{True}  \PY{c+c1}{\PYZsh{} Set palindrome Boolean to true }
        \PY{n}{mid\PYZus{}point}\PY{o}{=}\PY{n+nb}{int}\PY{p}{(}\PY{n+nb}{len}\PY{p}{(}\PY{n}{eval\PYZus{}str}\PY{p}{)}\PY{o}{/}\PY{l+m+mi}{2}\PY{p}{)} \PY{c+c1}{\PYZsh{} find midpoint of string to iterate over (truncated to deal with odd length)}
        \PY{n}{back\PYZus{}counter} \PY{o}{=} \PY{n+nb}{len}\PY{p}{(}\PY{n}{eval\PYZus{}str}\PY{p}{)}\PY{o}{\PYZhy{}}\PY{l+m+mi}{1} \PY{c+c1}{\PYZsh{} initialize the counter that works backward}
        \PY{k}{for} \PY{n}{counter} \PY{o+ow}{in} \PY{n+nb}{range}\PY{p}{(}\PY{n}{mid\PYZus{}point}\PY{p}{)} \PY{p}{:} \PY{c+c1}{\PYZsh{} Loop through the front half of the string}
            \PY{k}{if} \PY{n}{eval\PYZus{}str}\PY{p}{[}\PY{n}{counter}\PY{p}{]} \PY{o}{!=} \PY{n}{eval\PYZus{}str}\PY{p}{[}\PY{n}{back\PYZus{}counter}\PY{p}{]} \PY{p}{:}  \PY{c+c1}{\PYZsh{} Not a palindrome!}
                \PY{n}{is\PYZus{}palindrome}\PY{o}{=}\PY{k+kc}{False}
                \PY{k}{break}
            \PY{n}{back\PYZus{}counter} \PY{o}{\PYZhy{}}\PY{o}{=} \PY{l+m+mi}{1} \PY{c+c1}{\PYZsh{} Decrement the backwards counter}
    \PY{k}{return} \PY{n}{is\PYZus{}palindrome}   
\PY{c+c1}{\PYZsh{}}
\PY{c+c1}{\PYZsh{} Here are some tests to prove it works for odd and even length strings, the null string, and regardless of length}
\PY{c+c1}{\PYZsh{}}
\PY{n+nb}{print}\PY{p}{(}\PY{l+s+s2}{\PYZdq{}}\PY{l+s+s2}{\PYZsq{}}\PY{l+s+s2}{level}\PY{l+s+s2}{\PYZsq{}}\PY{l+s+s2}{:}\PY{l+s+s2}{\PYZdq{}}\PY{p}{,}\PY{n}{palindrome\PYZus{}for}\PY{p}{(}\PY{l+s+s2}{\PYZdq{}}\PY{l+s+s2}{level}\PY{l+s+s2}{\PYZdq{}}\PY{p}{)}\PY{p}{)}
\PY{n+nb}{print}\PY{p}{(}\PY{l+s+s2}{\PYZdq{}}\PY{l+s+s2}{\PYZsq{}}\PY{l+s+s2}{leveel}\PY{l+s+s2}{\PYZsq{}}\PY{l+s+s2}{:}\PY{l+s+s2}{\PYZdq{}}\PY{p}{,}\PY{n}{palindrome\PYZus{}for}\PY{p}{(}\PY{l+s+s2}{\PYZdq{}}\PY{l+s+s2}{leveel}\PY{l+s+s2}{\PYZdq{}}\PY{p}{)}\PY{p}{)}
\PY{n+nb}{print}\PY{p}{(}\PY{l+s+s2}{\PYZdq{}}\PY{l+s+s2}{\PYZsq{}}\PY{l+s+s2}{hannah}\PY{l+s+s2}{\PYZsq{}}\PY{l+s+s2}{:}\PY{l+s+s2}{\PYZdq{}}\PY{p}{,}\PY{n}{palindrome\PYZus{}for}\PY{p}{(}\PY{l+s+s2}{\PYZdq{}}\PY{l+s+s2}{hannah}\PY{l+s+s2}{\PYZdq{}}\PY{p}{)}\PY{p}{)}
\PY{n+nb}{print}\PY{p}{(}\PY{l+s+s2}{\PYZdq{}}\PY{l+s+s2}{\PYZsq{}}\PY{l+s+s2}{hanaah}\PY{l+s+s2}{\PYZsq{}}\PY{l+s+s2}{:}\PY{l+s+s2}{\PYZdq{}}\PY{p}{,}\PY{n}{palindrome\PYZus{}for}\PY{p}{(}\PY{l+s+s2}{\PYZdq{}}\PY{l+s+s2}{hanaah}\PY{l+s+s2}{\PYZdq{}}\PY{p}{)}\PY{p}{)}
\PY{n+nb}{print}\PY{p}{(}\PY{l+s+s2}{\PYZdq{}}\PY{l+s+s2}{If no string provided: }\PY{l+s+s2}{\PYZdq{}}\PY{p}{,}\PY{n}{palindrome\PYZus{}for}\PY{p}{(}\PY{l+s+s2}{\PYZdq{}}\PY{l+s+s2}{\PYZdq{}}\PY{p}{)}\PY{p}{)}
\PY{n+nb}{print}\PY{p}{(}\PY{l+s+s2}{\PYZdq{}}\PY{l+s+s2}{My favorite \PYZhy{} }\PY{l+s+s2}{\PYZsq{}}\PY{l+s+s2}{amanaplanacanalpanama}\PY{l+s+s2}{\PYZsq{}}\PY{l+s+s2}{:}\PY{l+s+s2}{\PYZdq{}}\PY{p}{,}\PY{n}{palindrome\PYZus{}for}\PY{p}{(}\PY{l+s+s2}{\PYZdq{}}\PY{l+s+s2}{amanaplanacanalpanama}\PY{l+s+s2}{\PYZdq{}}\PY{p}{)}\PY{p}{)}
\PY{n+nb}{print}\PY{p}{(}\PY{l+s+s2}{\PYZdq{}}\PY{l+s+s2}{Misspelled \PYZhy{} }\PY{l+s+s2}{\PYZsq{}}\PY{l+s+s2}{amanaplanaacanalpanama}\PY{l+s+s2}{\PYZsq{}}\PY{l+s+s2}{:}\PY{l+s+s2}{\PYZdq{}}\PY{p}{,}\PY{n}{palindrome\PYZus{}for}\PY{p}{(}\PY{l+s+s2}{\PYZdq{}}\PY{l+s+s2}{amanaplanaacanalpanama}\PY{l+s+s2}{\PYZdq{}}\PY{p}{)}\PY{p}{)}
\end{Verbatim}
\end{tcolorbox}

    \begin{Verbatim}[commandchars=\\\{\}]
'level': True
'leveel': False
'hannah': True
'hanaah': False
If no string provided:  False
My favorite - 'amanaplanacanalpanama': True
Misspelled - 'amanaplanaacanalpanama': False
    \end{Verbatim}

    \hypertarget{question-2}{%
\subsection{Question 2}\label{question-2}}

Write a function using a while-loop to determine if a string is a
palindrome. Your function should only have one argument.

    \begin{tcolorbox}[breakable, size=fbox, boxrule=1pt, pad at break*=1mm,colback=cellbackground, colframe=cellborder]
\prompt{In}{incolor}{11}{\boxspacing}
\begin{Verbatim}[commandchars=\\\{\}]
\PY{c+c1}{\PYZsh{} This code block / function uses the same logic as in Question 1.  The only difference is in the use of a While}
\PY{c+c1}{\PYZsh{} loop.  Using a While loop feels like better code hygiene. as the loop only runs while the palindrome Boolean}
\PY{c+c1}{\PYZsh{} is still True.  (The loop terminates once a mismatch has been found).  In order for this to function, we have to }
\PY{c+c1}{\PYZsh{} manually increment the forward counter in each loop run, and terminate the loop when EITHER it\PYZsq{}s not a palindrome OR}
\PY{c+c1}{\PYZsh{} we reached the midpoint of the string and so are done our work}
\PY{c+c1}{\PYZsh{}}
\PY{k}{def} \PY{n+nf}{palindrome\PYZus{}while}\PY{p}{(}\PY{n}{eval\PYZus{}str}\PY{p}{)}\PY{p}{:}
    \PY{c+c1}{\PYZsh{}}
    \PY{c+c1}{\PYZsh{} This function determines whether a string is a palindrome}
    \PY{c+c1}{\PYZsh{} Input:  eval\PYZus{}str = The string to be evaluated}
    \PY{c+c1}{\PYZsh{} Output:  The function returns a Boolean value True if eval\PYZus{}str is a palindrome, False if not}
    \PY{c+c1}{\PYZsh{}}
    \PY{k}{if} \PY{n+nb}{len}\PY{p}{(}\PY{n}{eval\PYZus{}str}\PY{p}{)}\PY{o}{==}\PY{l+m+mi}{0}\PY{p}{:} \PY{c+c1}{\PYZsh{} Check for null string and if so return False}
        \PY{n}{is\PYZus{}palindrome}\PY{o}{=}\PY{k+kc}{False}
    \PY{k}{else}\PY{p}{:}
        \PY{n}{is\PYZus{}palindrome}\PY{o}{=}\PY{k+kc}{True}  \PY{c+c1}{\PYZsh{} Initialize palindrome Boolean}
        \PY{n}{mid\PYZus{}point}\PY{o}{=}\PY{n+nb}{int}\PY{p}{(}\PY{n+nb}{len}\PY{p}{(}\PY{n}{eval\PYZus{}str}\PY{p}{)}\PY{o}{/}\PY{l+m+mi}{2}\PY{p}{)} \PY{c+c1}{\PYZsh{} Identify midpoint}
        \PY{n}{counter} \PY{o}{=} \PY{l+m+mi}{0} \PY{c+c1}{\PYZsh{} Initialize forward counter}
        \PY{n}{opp\PYZus{}counter} \PY{o}{=} \PY{n+nb}{len}\PY{p}{(}\PY{n}{eval\PYZus{}str}\PY{p}{)}\PY{o}{\PYZhy{}}\PY{l+m+mi}{1} \PY{c+c1}{\PYZsh{} Initialize backward counter}
        \PY{k}{while} \PY{n}{is\PYZus{}palindrome} \PY{o+ow}{and} \PY{n}{counter}\PY{o}{\PYZlt{}}\PY{o}{=}\PY{n}{mid\PYZus{}point}\PY{p}{:}  \PY{c+c1}{\PYZsh{} Loop until either hbit midpoint or fail palindrome test}
            \PY{k}{if} \PY{n}{eval\PYZus{}str}\PY{p}{[}\PY{n}{counter}\PY{p}{]} \PY{o}{!=} \PY{n}{eval\PYZus{}str}\PY{p}{[}\PY{n}{opp\PYZus{}counter}\PY{p}{]} \PY{p}{:} \PY{c+c1}{\PYZsh{} Not a palindrome}
                \PY{n}{is\PYZus{}palindrome}\PY{o}{=}\PY{k+kc}{False}   \PY{c+c1}{\PYZsh{} Will force the loop to end}
            \PY{k}{else}\PY{p}{:}
                \PY{n}{counter} \PY{o}{+}\PY{o}{=} \PY{l+m+mi}{1} \PY{c+c1}{\PYZsh{} Increment forward counter}
                \PY{n}{opp\PYZus{}counter} \PY{o}{\PYZhy{}}\PY{o}{=} \PY{l+m+mi}{1}  \PY{c+c1}{\PYZsh{} Decrement backward counter}
    \PY{k}{return} \PY{n}{is\PYZus{}palindrome}   

\PY{n+nb}{print}\PY{p}{(}\PY{l+s+s2}{\PYZdq{}}\PY{l+s+s2}{\PYZsq{}}\PY{l+s+s2}{level}\PY{l+s+s2}{\PYZsq{}}\PY{l+s+s2}{:}\PY{l+s+s2}{\PYZdq{}}\PY{p}{,}\PY{n}{palindrome\PYZus{}while}\PY{p}{(}\PY{l+s+s2}{\PYZdq{}}\PY{l+s+s2}{level}\PY{l+s+s2}{\PYZdq{}}\PY{p}{)}\PY{p}{)}
\PY{n+nb}{print}\PY{p}{(}\PY{l+s+s2}{\PYZdq{}}\PY{l+s+s2}{\PYZsq{}}\PY{l+s+s2}{leveel}\PY{l+s+s2}{\PYZsq{}}\PY{l+s+s2}{:}\PY{l+s+s2}{\PYZdq{}}\PY{p}{,}\PY{n}{palindrome\PYZus{}while}\PY{p}{(}\PY{l+s+s2}{\PYZdq{}}\PY{l+s+s2}{leveel}\PY{l+s+s2}{\PYZdq{}}\PY{p}{)}\PY{p}{)}
\PY{n+nb}{print}\PY{p}{(}\PY{l+s+s2}{\PYZdq{}}\PY{l+s+s2}{\PYZsq{}}\PY{l+s+s2}{hannah}\PY{l+s+s2}{\PYZsq{}}\PY{l+s+s2}{:}\PY{l+s+s2}{\PYZdq{}}\PY{p}{,}\PY{n}{palindrome\PYZus{}while}\PY{p}{(}\PY{l+s+s2}{\PYZdq{}}\PY{l+s+s2}{hannah}\PY{l+s+s2}{\PYZdq{}}\PY{p}{)}\PY{p}{)}
\PY{n+nb}{print}\PY{p}{(}\PY{l+s+s2}{\PYZdq{}}\PY{l+s+s2}{\PYZsq{}}\PY{l+s+s2}{hanaah}\PY{l+s+s2}{\PYZsq{}}\PY{l+s+s2}{:}\PY{l+s+s2}{\PYZdq{}}\PY{p}{,}\PY{n}{palindrome\PYZus{}while}\PY{p}{(}\PY{l+s+s2}{\PYZdq{}}\PY{l+s+s2}{hanaah}\PY{l+s+s2}{\PYZdq{}}\PY{p}{)}\PY{p}{)}
\PY{n+nb}{print}\PY{p}{(}\PY{l+s+s2}{\PYZdq{}}\PY{l+s+s2}{If no string provided: }\PY{l+s+s2}{\PYZdq{}}\PY{p}{,}\PY{n}{palindrome\PYZus{}while}\PY{p}{(}\PY{l+s+s2}{\PYZdq{}}\PY{l+s+s2}{\PYZdq{}}\PY{p}{)}\PY{p}{)}
\PY{n+nb}{print}\PY{p}{(}\PY{l+s+s2}{\PYZdq{}}\PY{l+s+s2}{My favorite \PYZhy{} }\PY{l+s+s2}{\PYZsq{}}\PY{l+s+s2}{amanaplanacanalpanama}\PY{l+s+s2}{\PYZsq{}}\PY{l+s+s2}{:}\PY{l+s+s2}{\PYZdq{}}\PY{p}{,}\PY{n}{palindrome\PYZus{}while}\PY{p}{(}\PY{l+s+s2}{\PYZdq{}}\PY{l+s+s2}{amanaplanacanalpanama}\PY{l+s+s2}{\PYZdq{}}\PY{p}{)}\PY{p}{)}
\PY{n+nb}{print}\PY{p}{(}\PY{l+s+s2}{\PYZdq{}}\PY{l+s+s2}{Misspelled \PYZhy{} }\PY{l+s+s2}{\PYZsq{}}\PY{l+s+s2}{amanaplanaacanalpanama}\PY{l+s+s2}{\PYZsq{}}\PY{l+s+s2}{:}\PY{l+s+s2}{\PYZdq{}}\PY{p}{,}\PY{n}{palindrome\PYZus{}while}\PY{p}{(}\PY{l+s+s2}{\PYZdq{}}\PY{l+s+s2}{amanaplanaacanalpanama}\PY{l+s+s2}{\PYZdq{}}\PY{p}{)}\PY{p}{)}
\end{Verbatim}
\end{tcolorbox}

    \begin{Verbatim}[commandchars=\\\{\}]
'level': True
'leveel': False
'hannah': True
'hanaah': False
If no string provided:  False
My favorite - 'amanaplanacanalpanama': True
Misspelled - 'amanaplanaacanalpanama': False
    \end{Verbatim}

    \hypertarget{question-3}{%
\subsection{Question 3}\label{question-3}}

Two Sum - Write a function named two\_sum() Given a vector of integers
nums and an integer target, return indices of the two numbers such that
they add up to target. You may assume that each input would have exactly
one solution, and you may not use the same element twice. You can return
the answer in any order.Use defaultdict and hash maps/tables to complete
this problem.

Example 1: Input: \texttt{nums\ =\ {[}2,7,11,15{]}}, target = 9 Output:
\texttt{{[}0,1{]}} Explanation: Because
\texttt{nums{[}0{]}\ +\ nums{[}1{]}\ ==\ 9}, we return
\texttt{{[}0,\ 1{]}}.

Example 2: Input: \texttt{nums\ =\ {[}3,2,4{]}},
\texttt{target\ =\ 6\ Output:\ {[}1,2{]}}

Example 3: Input: \texttt{nums\ =\ {[}3,3{]}},
\texttt{target\ =\ 6\ Output:\ {[}0,1{]}}

Constraints:
\texttt{2\ \textless{}=\ nums.length\ \textless{}=\ 104\ –109\ \textless{}=\ nums{[}i{]}\ \textless{}=\ 109\ –109\ \textless{}=\ target\ \textless{}=\ 109}
Only one valid answer exists.

    \begin{tcolorbox}[breakable, size=fbox, boxrule=1pt, pad at break*=1mm,colback=cellbackground, colframe=cellborder]
\prompt{In}{incolor}{12}{\boxspacing}
\begin{Verbatim}[commandchars=\\\{\}]
\PY{k+kn}{from} \PY{n+nn}{collections} \PY{k+kn}{import} \PY{n}{defaultdict}
\PY{c+c1}{\PYZsh{}    }
\PY{c+c1}{\PYZsh{} This function takes in a vector of values and a target value, and finds the}
\PY{c+c1}{\PYZsh{} first combination of indices such that the values in the vector at those indices}
\PY{c+c1}{\PYZsh{} adds up to the target value}
\PY{c+c1}{\PYZsh{}}
\PY{c+c1}{\PYZsh{} Inputs: nums\PYZus{}vector = The vector of values to be searched}
\PY{c+c1}{\PYZsh{} target = The target sum being sought}
\PY{c+c1}{\PYZsh{}}
\PY{c+c1}{\PYZsh{} Output: The function returns a vector (list) containing the two indexes whose values add up to target}
\PY{c+c1}{\PYZsh{} The function will print an error message and return null if any of the argument constraints are violated }
\PY{k}{def} \PY{n+nf}{two\PYZus{}sum}\PY{p}{(}\PY{n}{nums\PYZus{}vector}\PY{p}{,}\PY{n}{target}\PY{p}{)}\PY{p}{:}
\PY{c+c1}{\PYZsh{} First check for constraint violations in the arguments submitted}
    \PY{n}{argumentsOK}\PY{o}{=}\PY{k+kc}{True}
    \PY{k}{if} \PY{n+nb}{len}\PY{p}{(}\PY{n}{nums\PYZus{}vector}\PY{p}{)} \PY{o}{\PYZlt{}} \PY{l+m+mi}{2}\PY{p}{:}
        \PY{n+nb}{print}\PY{p}{(}\PY{l+s+s2}{\PYZdq{}}\PY{l+s+s2}{ERROR: Number vector must have at least two values}\PY{l+s+s2}{\PYZdq{}}\PY{p}{)}
        \PY{n}{argumentsOK} \PY{o}{=} \PY{k+kc}{False}
    \PY{k}{if} \PY{n+nb}{len}\PY{p}{(}\PY{n}{nums\PYZus{}vector}\PY{p}{)}\PY{o}{\PYZgt{}}\PY{l+m+mi}{104} \PY{p}{:}
        \PY{n+nb}{print}\PY{p}{(}\PY{l+s+s2}{\PYZdq{}}\PY{l+s+s2}{ERROR: Number vector cannot exceed 104 values}\PY{l+s+s2}{\PYZdq{}}\PY{p}{)}
        \PY{n}{argumentsOK} \PY{o}{=} \PY{k+kc}{False}
    \PY{k}{if} \PY{n+nb}{min}\PY{p}{(}\PY{n}{nums\PYZus{}vector}\PY{p}{)} \PY{o}{\PYZlt{}} \PY{o}{\PYZhy{}}\PY{l+m+mi}{109} \PY{p}{:}
        \PY{n+nb}{print}\PY{p}{(}\PY{l+s+s2}{\PYZdq{}}\PY{l+s+s2}{ERROR: All values in vector must be \PYZgt{}= \PYZhy{}109}\PY{l+s+s2}{\PYZdq{}}\PY{p}{)}
        \PY{n}{argumentsOK} \PY{o}{=} \PY{k+kc}{False}
    \PY{k}{if} \PY{n+nb}{max}\PY{p}{(}\PY{n}{nums\PYZus{}vector}\PY{p}{)} \PY{o}{\PYZgt{}} \PY{l+m+mi}{109} \PY{p}{:}
        \PY{n+nb}{print}\PY{p}{(}\PY{l+s+s2}{\PYZdq{}}\PY{l+s+s2}{ERROR: All values in vector must be \PYZlt{}= 109}\PY{l+s+s2}{\PYZdq{}}\PY{p}{)}
        \PY{n}{argumentsOK} \PY{o}{=} \PY{k+kc}{False}
    \PY{k}{if} \PY{n}{target} \PY{o}{\PYZlt{}} \PY{o}{\PYZhy{}}\PY{l+m+mi}{109} \PY{p}{:}
        \PY{n+nb}{print}\PY{p}{(}\PY{l+s+s2}{\PYZdq{}}\PY{l+s+s2}{ERROR: Target sum must be \PYZgt{}= \PYZhy{}109}\PY{l+s+s2}{\PYZdq{}}\PY{p}{)}
        \PY{n}{argumentsOK} \PY{o}{=} \PY{k+kc}{False}
    \PY{k}{if} \PY{n}{target} \PY{o}{\PYZgt{}} \PY{l+m+mi}{109} \PY{p}{:}
        \PY{n+nb}{print}\PY{p}{(}\PY{l+s+s2}{\PYZdq{}}\PY{l+s+s2}{ERROR: Target sum must be \PYZlt{}= 109}\PY{l+s+s2}{\PYZdq{}}\PY{p}{)}
        \PY{n}{argumentsOK} \PY{o}{=} \PY{k+kc}{False}
\PY{c+c1}{\PYZsh{} If any error was triggered, abort the function}
    \PY{k}{if} \PY{o+ow}{not} \PY{n}{argumentsOK}\PY{p}{:} 
        \PY{n+nb}{print}\PY{p}{(}\PY{l+s+s2}{\PYZdq{}}\PY{l+s+s2}{Function Aborted Due to bad argument(s) above}\PY{l+s+s2}{\PYZdq{}}\PY{p}{)}
        \PY{k}{return} 
    \PY{k}{else}\PY{p}{:}
        \PY{k}{def} \PY{n+nf}{def\PYZus{}value}\PY{p}{(}\PY{p}{)}\PY{p}{:}
            \PY{k}{return} \PY{l+s+s2}{\PYZdq{}}\PY{l+s+s2}{Not Present}\PY{l+s+s2}{\PYZdq{}} \PY{c+c1}{\PYZsh{} Set default value \PYZhy{} this will be used to flag whether a sum complement is founbd                  }
        \PY{n}{hash\PYZus{}table}\PY{o}{=}\PY{n}{defaultdict}\PY{p}{(}\PY{n}{def\PYZus{}value}\PY{p}{)} \PY{c+c1}{\PYZsh{} Create null hashtable}
\PY{c+c1}{\PYZsh{} Populate hashtable:  Keys are each value in the vector, values are the index where that value occurs }
\PY{c+c1}{\PYZsh{} in the vector (the counter) }
        \PY{k}{for} \PY{n}{counter1} \PY{o+ow}{in} \PY{n+nb}{range}\PY{p}{(}\PY{n+nb}{len}\PY{p}{(}\PY{n}{nums\PYZus{}vector}\PY{p}{)}\PY{p}{)}\PY{p}{:}   
            \PY{n}{hash\PYZus{}table}\PY{p}{[}\PY{n}{nums\PYZus{}vector}\PY{p}{[}\PY{n}{counter1}\PY{p}{]}\PY{p}{]} \PY{o}{=} \PY{n}{counter1}
\PY{c+c1}{\PYZsh{} Now we go through the list again; For each element, search the hashtable for its \PYZdq{}complement\PYZdq{} that when added to it,}
\PY{c+c1}{\PYZsh{} equates to target.  If found, and it isn\PYZsq{}t adding an element to itself, we have successfully identified the pair}
        \PY{n}{counter2}\PY{o}{=}\PY{l+m+mi}{0}
        \PY{n}{match\PYZus{}found}\PY{o}{=}\PY{k+kc}{False}
        \PY{k}{while} \PY{n}{counter2} \PY{o}{\PYZlt{}}\PY{o}{=} \PY{n+nb}{len}\PY{p}{(}\PY{n}{nums\PYZus{}vector}\PY{p}{)} \PY{o+ow}{and} \PY{o+ow}{not} \PY{n}{match\PYZus{}found}\PY{p}{:}
            \PY{n}{search\PYZus{}key} \PY{o}{=} \PY{n}{target}\PY{o}{\PYZhy{}}\PY{n}{nums\PYZus{}vector}\PY{p}{[}\PY{n}{counter2}\PY{p}{]} \PY{c+c1}{\PYZsh{} this is the complement we\PYZsq{}re looking for}
            \PY{n}{match\PYZus{}location} \PY{o}{=} \PY{n}{hash\PYZus{}table}\PY{p}{[}\PY{n}{search\PYZus{}key}\PY{p}{]} \PY{c+c1}{\PYZsh{} Retrieve the value associated with that key, which will be}
                                                    \PY{c+c1}{\PYZsh{} equal to the default value if it isn\PYZsq{}t found}
            \PY{k}{if} \PY{n}{match\PYZus{}location} \PY{o}{!=} \PY{n}{def\PYZus{}value}\PY{p}{(}\PY{p}{)} \PY{o+ow}{and} \PY{n}{match\PYZus{}location} \PY{o}{!=} \PY{n}{counter2}\PY{p}{:}  \PY{c+c1}{\PYZsh{} Found a non\PYZhy{}duplicative match!}
                \PY{n}{match} \PY{o}{=} \PY{p}{[}\PY{n}{counter2}\PY{p}{,}\PY{n}{match\PYZus{}location}\PY{p}{]}
                \PY{n}{match\PYZus{}found}\PY{o}{=}\PY{k+kc}{True}
            \PY{k}{else}\PY{p}{:}
                \PY{n}{counter2} \PY{o}{+}\PY{o}{=} \PY{l+m+mi}{1}
    \PY{k}{return} \PY{n}{match}

\PY{c+c1}{\PYZsh{} Prove that error checking works}
\PY{n}{nums\PYZus{}vector}\PY{o}{=}\PY{p}{[}\PY{l+m+mi}{1}\PY{p}{]}
\PY{n}{two\PYZus{}sum}\PY{p}{(}\PY{n}{nums\PYZus{}vector}\PY{p}{,}\PY{l+m+mi}{2}\PY{p}{)} \PY{c+c1}{\PYZsh{}nums\PYZus{}vector too short}
\PY{n}{nums\PYZus{}vector}\PY{o}{=}\PY{n+nb}{list}\PY{p}{(}\PY{n+nb}{range}\PY{p}{(}\PY{l+m+mi}{1}\PY{p}{,}\PY{l+m+mi}{106}\PY{p}{)}\PY{p}{)}
\PY{n}{two\PYZus{}sum}\PY{p}{(}\PY{n}{nums\PYZus{}vector}\PY{p}{,}\PY{l+m+mi}{2}\PY{p}{)} \PY{c+c1}{\PYZsh{}nums\PYZus{}vector too long}
\PY{n}{nums\PYZus{}vector}\PY{o}{=}\PY{n+nb}{list}\PY{p}{(}\PY{n+nb}{range}\PY{p}{(}\PY{o}{\PYZhy{}}\PY{l+m+mi}{110}\PY{p}{,}\PY{o}{\PYZhy{}}\PY{l+m+mi}{10}\PY{p}{)}\PY{p}{)}
\PY{n}{two\PYZus{}sum}\PY{p}{(}\PY{n}{nums\PYZus{}vector}\PY{p}{,}\PY{l+m+mi}{2}\PY{p}{)} \PY{c+c1}{\PYZsh{}lowest nums\PYZus{}vector value not in bounds}
\PY{n}{nums\PYZus{}vector}\PY{o}{=}\PY{n+nb}{list}\PY{p}{(}\PY{n+nb}{range}\PY{p}{(}\PY{l+m+mi}{100}\PY{p}{,}\PY{l+m+mi}{111}\PY{p}{)}\PY{p}{)}
\PY{n}{two\PYZus{}sum}\PY{p}{(}\PY{n}{nums\PYZus{}vector}\PY{p}{,}\PY{l+m+mi}{2}\PY{p}{)} \PY{c+c1}{\PYZsh{}highest nums\PYZus{}vector value not in bounds}
\PY{n}{nums\PYZus{}vector}\PY{o}{=}\PY{n+nb}{list}\PY{p}{(}\PY{n+nb}{range}\PY{p}{(}\PY{l+m+mi}{1}\PY{p}{,}\PY{l+m+mi}{101}\PY{p}{)}\PY{p}{)}
\PY{n}{two\PYZus{}sum}\PY{p}{(}\PY{n}{nums\PYZus{}vector}\PY{p}{,}\PY{o}{\PYZhy{}}\PY{l+m+mi}{110}\PY{p}{)} \PY{c+c1}{\PYZsh{}target below lower bound}
\PY{n}{two\PYZus{}sum}\PY{p}{(}\PY{n}{nums\PYZus{}vector}\PY{p}{,}\PY{l+m+mi}{110}\PY{p}{)} \PY{c+c1}{\PYZsh{}target above higher bound}
\PY{n}{nums\PYZus{}vector}\PY{o}{=}\PY{n+nb}{list}\PY{p}{(}\PY{n+nb}{range}\PY{p}{(}\PY{o}{\PYZhy{}}\PY{l+m+mi}{120}\PY{p}{,}\PY{l+m+mi}{120}\PY{p}{)}\PY{p}{)}
\PY{n}{target}\PY{o}{=}\PY{l+m+mi}{200}
\PY{n}{two\PYZus{}sum}\PY{p}{(}\PY{n}{nums\PYZus{}vector}\PY{p}{,}\PY{n}{target}\PY{p}{)} \PY{c+c1}{\PYZsh{}multiple constraint violations}

\PY{c+c1}{\PYZsh{} Now run function on test cases}
\PY{n}{nums\PYZus{}vector} \PY{o}{=} \PY{p}{[}\PY{l+m+mi}{2}\PY{p}{,}\PY{l+m+mi}{7}\PY{p}{,}\PY{l+m+mi}{11}\PY{p}{,}\PY{l+m+mi}{15}\PY{p}{]}
\PY{n}{target} \PY{o}{=} \PY{l+m+mi}{9}
\PY{n+nb}{print}\PY{p}{(}\PY{n}{two\PYZus{}sum}\PY{p}{(}\PY{n}{nums\PYZus{}vector}\PY{p}{,}\PY{n}{target}\PY{p}{)}\PY{p}{)}
\PY{n}{nums\PYZus{}vector} \PY{o}{=} \PY{p}{[}\PY{l+m+mi}{3}\PY{p}{,}\PY{l+m+mi}{2}\PY{p}{,}\PY{l+m+mi}{4}\PY{p}{]}
\PY{n}{target} \PY{o}{=} \PY{l+m+mi}{6}
\PY{n+nb}{print}\PY{p}{(}\PY{n}{two\PYZus{}sum}\PY{p}{(}\PY{n}{nums\PYZus{}vector}\PY{p}{,}\PY{n}{target}\PY{p}{)}\PY{p}{)}
\PY{n}{nums\PYZus{}vector} \PY{o}{=} \PY{p}{[}\PY{l+m+mi}{3}\PY{p}{,}\PY{l+m+mi}{3}\PY{p}{]}
\PY{n}{target} \PY{o}{=} \PY{l+m+mi}{6}
\PY{n+nb}{print}\PY{p}{(}\PY{n}{two\PYZus{}sum}\PY{p}{(}\PY{n}{nums\PYZus{}vector}\PY{p}{,}\PY{n}{target}\PY{p}{)}\PY{p}{)}
\PY{n}{nums\PYZus{}vector} \PY{o}{=} \PY{p}{[}\PY{l+m+mi}{1}\PY{p}{,}\PY{l+m+mi}{3}\PY{p}{,}\PY{l+m+mi}{6}\PY{p}{,}\PY{l+m+mi}{10}\PY{p}{,}\PY{l+m+mi}{15}\PY{p}{,}\PY{l+m+mi}{21}\PY{p}{,}\PY{l+m+mi}{28}\PY{p}{,}\PY{l+m+mi}{36}\PY{p}{,}\PY{l+m+mi}{45}\PY{p}{,}\PY{l+m+mi}{55}\PY{p}{,}\PY{l+m+mi}{66}\PY{p}{,}\PY{l+m+mi}{78}\PY{p}{,}\PY{l+m+mi}{91}\PY{p}{,}\PY{l+m+mi}{105}\PY{p}{]}
\PY{n}{target}\PY{o}{=}\PY{l+m+mi}{97}
\PY{n+nb}{print}\PY{p}{(}\PY{n}{two\PYZus{}sum}\PY{p}{(}\PY{n}{nums\PYZus{}vector}\PY{p}{,}\PY{n}{target}\PY{p}{)}\PY{p}{)}
\PY{n}{target}\PY{o}{=}\PY{l+m+mi}{73}
\PY{n+nb}{print}\PY{p}{(}\PY{n}{two\PYZus{}sum}\PY{p}{(}\PY{n}{nums\PYZus{}vector}\PY{p}{,}\PY{n}{target}\PY{p}{)}\PY{p}{)}
\end{Verbatim}
\end{tcolorbox}

    \begin{Verbatim}[commandchars=\\\{\}]
ERROR: Number vector must have at least two values
Function Aborted Due to bad argument(s) above
ERROR: Number vector cannot exceed 104 values
Function Aborted Due to bad argument(s) above
ERROR: All values in vector must be >= -109
Function Aborted Due to bad argument(s) above
ERROR: All values in vector must be <= 109
Function Aborted Due to bad argument(s) above
ERROR: Target sum must be >= -109
Function Aborted Due to bad argument(s) above
ERROR: Target sum must be <= 109
Function Aborted Due to bad argument(s) above
ERROR: Number vector cannot exceed 104 values
ERROR: All values in vector must be >= -109
ERROR: All values in vector must be <= 109
ERROR: Target sum must be <= 109
Function Aborted Due to bad argument(s) above
[0, 1]
[1, 2]
[0, 1]
[2, 12]
[6, 8]
    \end{Verbatim}

    \hypertarget{question-4}{%
\subsection{Question 4}\label{question-4}}

How is a negative index used in Python? Show an example

    \begin{tcolorbox}[breakable, size=fbox, boxrule=1pt, pad at break*=1mm,colback=cellbackground, colframe=cellborder]
\prompt{In}{incolor}{14}{\boxspacing}
\begin{Verbatim}[commandchars=\\\{\}]
\PY{c+c1}{\PYZsh{} Wow this whole lab (well at least Questions 1 and 2) was a setup leading to this!   Negative indexing lets you start}
\PY{c+c1}{\PYZsh{} from the end rather than the beginning of a list, array etc.  You could use that for example to extract the right\PYZhy{}}
\PY{c+c1}{\PYZsh{} most N characters from a string \PYZhy{} for example, if a location is listed as \PYZdq{}City, ST\PYZdq{} like \PYZdq{}Boston, MA\PYZdq{} you could}
\PY{c+c1}{\PYZsh{} isolate the state as the last two characters in the string using negative indexing.}
\PY{c+c1}{\PYZsh{}}
\PY{c+c1}{\PYZsh{} Why was this a setup? Because we don\PYZsq{}t need to write a complicated for loop or while loop to identify whether}
\PY{c+c1}{\PYZsh{} a string is a palindrome!  We can just use this negative indexing to return the string in reverse order and }
\PY{c+c1}{\PYZsh{} compare that to the original.}
\PY{c+c1}{\PYZsh{} }
\PY{c+c1}{\PYZsh{} Some syntax:  slicing the string is string[start:stop:step];  If you leave start and stop blank and use \PYZhy{}1 for}
\PY{c+c1}{\PYZsh{} step it literally gives you the string in reverse:}
\PY{n}{test\PYZus{}str}\PY{o}{=}\PY{l+s+s2}{\PYZdq{}}\PY{l+s+s2}{Raymond}\PY{l+s+s2}{\PYZdq{}}
\PY{n}{backwards}\PY{o}{=}\PY{n}{test\PYZus{}str}\PY{p}{[}\PY{p}{:}\PY{p}{:}\PY{o}{\PYZhy{}}\PY{l+m+mi}{1}\PY{p}{]}
\PY{n+nb}{print}\PY{p}{(}\PY{n}{test\PYZus{}str}\PY{p}{,}\PY{l+s+s2}{\PYZdq{}}\PY{l+s+s2}{backwards is}\PY{l+s+s2}{\PYZdq{}}\PY{p}{,}\PY{n}{backwards}\PY{p}{)}
\PY{c+c1}{\PYZsh{}}
\PY{c+c1}{\PYZsh{} Knowing this, I can very easily test for palindromes!}
\PY{k}{def} \PY{n+nf}{palindrome\PYZus{}easy}\PY{p}{(}\PY{n}{eval\PYZus{}str}\PY{p}{)}\PY{p}{:}
    \PY{k}{if} \PY{n+nb}{len}\PY{p}{(}\PY{n}{eval\PYZus{}str}\PY{p}{)}\PY{o}{==}\PY{l+m+mi}{0}\PY{p}{:}
        \PY{k}{return} \PY{k+kc}{False}
    \PY{k}{else}\PY{p}{:}
        \PY{n}{backwards} \PY{o}{=} \PY{n}{eval\PYZus{}str}\PY{p}{[}\PY{p}{:}\PY{p}{:}\PY{o}{\PYZhy{}}\PY{l+m+mi}{1}\PY{p}{]}
        \PY{n}{is\PYZus{}palindrome}\PY{o}{=} \PY{n}{backwards} \PY{o}{==} \PY{n}{eval\PYZus{}str}
    \PY{k}{return} \PY{n}{is\PYZus{}palindrome}

\PY{n+nb}{print}\PY{p}{(}\PY{l+s+s2}{\PYZdq{}}\PY{l+s+s2}{\PYZsq{}}\PY{l+s+s2}{level}\PY{l+s+s2}{\PYZsq{}}\PY{l+s+s2}{:}\PY{l+s+s2}{\PYZdq{}}\PY{p}{,}\PY{n}{palindrome\PYZus{}easy}\PY{p}{(}\PY{l+s+s2}{\PYZdq{}}\PY{l+s+s2}{level}\PY{l+s+s2}{\PYZdq{}}\PY{p}{)}\PY{p}{)}
\PY{n+nb}{print}\PY{p}{(}\PY{l+s+s2}{\PYZdq{}}\PY{l+s+s2}{\PYZsq{}}\PY{l+s+s2}{leveel}\PY{l+s+s2}{\PYZsq{}}\PY{l+s+s2}{:}\PY{l+s+s2}{\PYZdq{}}\PY{p}{,}\PY{n}{palindrome\PYZus{}easy}\PY{p}{(}\PY{l+s+s2}{\PYZdq{}}\PY{l+s+s2}{leveel}\PY{l+s+s2}{\PYZdq{}}\PY{p}{)}\PY{p}{)}
\PY{n+nb}{print}\PY{p}{(}\PY{l+s+s2}{\PYZdq{}}\PY{l+s+s2}{\PYZsq{}}\PY{l+s+s2}{hannah}\PY{l+s+s2}{\PYZsq{}}\PY{l+s+s2}{:}\PY{l+s+s2}{\PYZdq{}}\PY{p}{,}\PY{n}{palindrome\PYZus{}easy}\PY{p}{(}\PY{l+s+s2}{\PYZdq{}}\PY{l+s+s2}{hannah}\PY{l+s+s2}{\PYZdq{}}\PY{p}{)}\PY{p}{)}
\PY{n+nb}{print}\PY{p}{(}\PY{l+s+s2}{\PYZdq{}}\PY{l+s+s2}{\PYZsq{}}\PY{l+s+s2}{hanaah}\PY{l+s+s2}{\PYZsq{}}\PY{l+s+s2}{:}\PY{l+s+s2}{\PYZdq{}}\PY{p}{,}\PY{n}{palindrome\PYZus{}easy}\PY{p}{(}\PY{l+s+s2}{\PYZdq{}}\PY{l+s+s2}{hanaah}\PY{l+s+s2}{\PYZdq{}}\PY{p}{)}\PY{p}{)}
\PY{n+nb}{print}\PY{p}{(}\PY{l+s+s2}{\PYZdq{}}\PY{l+s+s2}{If no string provided: }\PY{l+s+s2}{\PYZdq{}}\PY{p}{,}\PY{n}{palindrome\PYZus{}easy}\PY{p}{(}\PY{l+s+s2}{\PYZdq{}}\PY{l+s+s2}{\PYZdq{}}\PY{p}{)}\PY{p}{)}
\PY{n+nb}{print}\PY{p}{(}\PY{l+s+s2}{\PYZdq{}}\PY{l+s+s2}{My favorite \PYZhy{} }\PY{l+s+s2}{\PYZsq{}}\PY{l+s+s2}{amanaplanacanalpanama}\PY{l+s+s2}{\PYZsq{}}\PY{l+s+s2}{:}\PY{l+s+s2}{\PYZdq{}}\PY{p}{,}\PY{n}{palindrome\PYZus{}easy}\PY{p}{(}\PY{l+s+s2}{\PYZdq{}}\PY{l+s+s2}{amanaplanacanalpanama}\PY{l+s+s2}{\PYZdq{}}\PY{p}{)}\PY{p}{)}
\PY{n+nb}{print}\PY{p}{(}\PY{l+s+s2}{\PYZdq{}}\PY{l+s+s2}{Misspelled \PYZhy{} }\PY{l+s+s2}{\PYZsq{}}\PY{l+s+s2}{amanaplanaacanalpanama}\PY{l+s+s2}{\PYZsq{}}\PY{l+s+s2}{:}\PY{l+s+s2}{\PYZdq{}}\PY{p}{,}\PY{n}{palindrome\PYZus{}easy}\PY{p}{(}\PY{l+s+s2}{\PYZdq{}}\PY{l+s+s2}{amanaplanaacanalpanama}\PY{l+s+s2}{\PYZdq{}}\PY{p}{)}\PY{p}{)}
\end{Verbatim}
\end{tcolorbox}

    \begin{Verbatim}[commandchars=\\\{\}]
Raymond backwards is dnomyaR
'level': True
'leveel': False
'hannah': True
'hanaah': False
If no string provided:  False
My favorite - 'amanaplanacanalpanama': True
Misspelled - 'amanaplanaacanalpanama': False
    \end{Verbatim}

    \hypertarget{question-5}{%
\subsection{Question 5}\label{question-5}}

Check if two given strings are isomorphic to each other. Two strings
str1 and str2 are called isomorphic if there is a one-to-one mapping
possible for every character of str1 to every character of str2. And all
occurrences of every character in `str1' map to the same character in
`str2'.

\texttt{Input:\ \ str1\ =\ "aab",\ str2\ =\ "xxy"}

\texttt{Output:\ True}

\texttt{\textquotesingle{}a\textquotesingle{}\ is\ mapped\ to\ \textquotesingle{}x\textquotesingle{}\ and\ \textquotesingle{}b\textquotesingle{}\ is\ mapped\ to\ \textquotesingle{}y\textquotesingle{}.}

\texttt{Input:\ \ str1\ =\ "aab",\ str2\ =\ "xyz"}

\texttt{Output:\ False}

\texttt{One\ occurrence\ of\ \textquotesingle{}a\textquotesingle{}\ in\ str1\ has\ \textquotesingle{}x\textquotesingle{}\ in\ str2\ and\ other\ occurrence\ of\ \textquotesingle{}a\textquotesingle{}\ has\ \textquotesingle{}y\textquotesingle{}.}

A Simple Solution is to consider every character of `str1' and check if
all occurrences of it map to the same character in `str2'. The time
complexity of this solution is O(n*n).

An Efficient Solution can solve this problem in O(n) time. The idea is
to create an array to store mappings of processed characters.

    \begin{tcolorbox}[breakable, size=fbox, boxrule=1pt, pad at break*=1mm,colback=cellbackground, colframe=cellborder]
\prompt{In}{incolor}{18}{\boxspacing}
\begin{Verbatim}[commandchars=\\\{\}]
\PY{c+c1}{\PYZsh{} This lends itself well to using a hashtable, with the key being character in str1 and the value being the character}
\PY{c+c1}{\PYZsh{} it maps to in str2.  This is an easy construct: }
\PY{c+c1}{\PYZsh{} \PYZhy{} Loop through the chracters in str1.  }
\PY{c+c1}{\PYZsh{} \PYZhy{} For each character, check to see if it is already in the hashtable}
\PY{c+c1}{\PYZsh{} \PYZhy{} If it is, and the matching character value (from str2) doesn\PYZsq{}t match what we have, then it\PYZsq{}s a fail!}
\PY{c+c1}{\PYZsh{} \PYZhy{} Otherwise, add this character and its matching character from str2 into the hashtable}
\PY{c+c1}{\PYZsh{}   (If the character pair is already in the hashtable, it won\PYZsq{}t be duplicated \PYZhy{} maybe it\PYZsq{}s lazy programming}
\PY{c+c1}{\PYZsh{}    to take advantage of this, but it works just fine) \PYZhy{} I could test for this first if necessary but result is same}
\PY{c+c1}{\PYZsh{}}
\PY{c+c1}{\PYZsh{} So the idea is we build a hashtable of unique (char1,char2) combos but as soon as we find an entry with same char1}
\PY{c+c1}{\PYZsh{} but different char2, we know we have failed the isomorphic test}
\PY{c+c1}{\PYZsh{} From an algorithm performance standpoint, this will take at MOST O(n) time, and then only if we have to evaluate}
\PY{c+c1}{\PYZsh{} every character in str1.}
\PY{k}{def} \PY{n+nf}{isomorphic}\PY{p}{(}\PY{n}{str1}\PY{p}{,} \PY{n}{str2}\PY{p}{)}\PY{p}{:}
    \PY{k}{def} \PY{n+nf}{def\PYZus{}value}\PY{p}{(}\PY{p}{)}\PY{p}{:}
        \PY{k}{return} \PY{l+s+s2}{\PYZdq{}}\PY{l+s+s2}{Not Present}\PY{l+s+s2}{\PYZdq{}}      
    \PY{k}{if} \PY{n+nb}{len}\PY{p}{(}\PY{n}{str1}\PY{p}{)} \PY{o}{!=} \PY{n+nb}{len}\PY{p}{(}\PY{n}{str2}\PY{p}{)}\PY{p}{:}  \PY{c+c1}{\PYZsh{}if strings aren\PYZsq{}t same length it\PYZsq{}s an automatic fail}
        \PY{k}{return} \PY{k+kc}{False}
    \PY{k}{else}\PY{p}{:}
        \PY{n}{char\PYZus{}maps} \PY{o}{=} \PY{n}{defaultdict}\PY{p}{(}\PY{n}{def\PYZus{}value}\PY{p}{)}  \PY{c+c1}{\PYZsh{} set null hashmap}
        \PY{n}{counter}\PY{o}{=}\PY{l+m+mi}{0}
        \PY{n}{is\PYZus{}isomorphic} \PY{o}{=} \PY{k+kc}{True} 
        \PY{k}{while} \PY{n}{counter} \PY{o}{\PYZlt{}} \PY{n+nb}{len}\PY{p}{(}\PY{n}{str1}\PY{p}{)} \PY{o+ow}{and} \PY{n}{is\PYZus{}isomorphic}\PY{p}{:} \PY{c+c1}{\PYZsh{} loop through all characters in str1 until failure or done}
            \PY{k}{if} \PY{n}{char\PYZus{}maps}\PY{p}{[}\PY{n}{str1}\PY{p}{[}\PY{n}{counter}\PY{p}{]}\PY{p}{]} \PY{o}{!=} \PY{n}{def\PYZus{}value}\PY{p}{(}\PY{p}{)} \PY{o+ow}{and} \PY{n}{char\PYZus{}maps}\PY{p}{[}\PY{n}{str1}\PY{p}{[}\PY{n}{counter}\PY{p}{]}\PY{p}{]} \PY{o}{!=} \PY{n}{str2}\PY{p}{[}\PY{n}{counter}\PY{p}{]}\PY{p}{:} \PY{c+c1}{\PYZsh{} Check to see}
                \PY{c+c1}{\PYZsh{}if in the list already, with a different matching char2}
                \PY{n}{is\PYZus{}isomorphic} \PY{o}{=} \PY{k+kc}{False}
            \PY{k}{else}\PY{p}{:}
                \PY{n}{char\PYZus{}maps}\PY{p}{[}\PY{n}{str1}\PY{p}{[}\PY{n}{counter}\PY{p}{]}\PY{p}{]}\PY{o}{=}\PY{n}{str2}\PY{p}{[}\PY{n}{counter}\PY{p}{]} \PY{c+c1}{\PYZsh{} if not, add it to hashmap (knowing dupes are discarded)}
                \PY{n}{counter} \PY{o}{+}\PY{o}{=}\PY{l+m+mi}{1}
    \PY{k}{return} \PY{n}{is\PYZus{}isomorphic}
\PY{c+c1}{\PYZsh{} Let\PYZsq{}s test it shall we?}
\PY{n}{str1}\PY{o}{=}\PY{l+s+s2}{\PYZdq{}}\PY{l+s+s2}{aab}\PY{l+s+s2}{\PYZdq{}}
\PY{n}{str2}\PY{o}{=}\PY{l+s+s2}{\PYZdq{}}\PY{l+s+s2}{xxy}\PY{l+s+s2}{\PYZdq{}}
\PY{n+nb}{print}\PY{p}{(}\PY{n}{isomorphic}\PY{p}{(}\PY{n}{str1}\PY{p}{,} \PY{n}{str2}\PY{p}{)}\PY{p}{)}
\PY{n}{str1}\PY{o}{=}\PY{l+s+s2}{\PYZdq{}}\PY{l+s+s2}{aab}\PY{l+s+s2}{\PYZdq{}}
\PY{n}{str2}\PY{o}{=}\PY{l+s+s2}{\PYZdq{}}\PY{l+s+s2}{xyz}\PY{l+s+s2}{\PYZdq{}}
\PY{n+nb}{print}\PY{p}{(}\PY{n}{isomorphic}\PY{p}{(}\PY{n}{str1}\PY{p}{,} \PY{n}{str2}\PY{p}{)}\PY{p}{)}
\PY{n}{str1}\PY{o}{=}\PY{l+s+s2}{\PYZdq{}}\PY{l+s+s2}{aabbccddabcddcba}\PY{l+s+s2}{\PYZdq{}}
\PY{n}{str2}\PY{o}{=}\PY{l+s+s2}{\PYZdq{}}\PY{l+s+s2}{wwxxyyzzwxyzzyxw}\PY{l+s+s2}{\PYZdq{}}
\PY{n+nb}{print}\PY{p}{(}\PY{n}{isomorphic}\PY{p}{(}\PY{n}{str1}\PY{p}{,} \PY{n}{str2}\PY{p}{)}\PY{p}{)}
\PY{n}{str1}\PY{o}{=}\PY{l+s+s2}{\PYZdq{}}\PY{l+s+s2}{aabbccddabcddcba}\PY{l+s+s2}{\PYZdq{}}
\PY{n}{str2}\PY{o}{=}\PY{l+s+s2}{\PYZdq{}}\PY{l+s+s2}{wwxxyyzzwxyzayxw}\PY{l+s+s2}{\PYZdq{}}
\PY{n+nb}{print}\PY{p}{(}\PY{n}{isomorphic}\PY{p}{(}\PY{n}{str1}\PY{p}{,} \PY{n}{str2}\PY{p}{)}\PY{p}{)}
\end{Verbatim}
\end{tcolorbox}

    \begin{Verbatim}[commandchars=\\\{\}]
True
False
True
False
    \end{Verbatim}

    \begin{tcolorbox}[breakable, size=fbox, boxrule=1pt, pad at break*=1mm,colback=cellbackground, colframe=cellborder]
\prompt{In}{incolor}{ }{\boxspacing}
\begin{Verbatim}[commandchars=\\\{\}]

\end{Verbatim}
\end{tcolorbox}

    \begin{tcolorbox}[breakable, size=fbox, boxrule=1pt, pad at break*=1mm,colback=cellbackground, colframe=cellborder]
\prompt{In}{incolor}{ }{\boxspacing}
\begin{Verbatim}[commandchars=\\\{\}]

\end{Verbatim}
\end{tcolorbox}


    % Add a bibliography block to the postdoc
    
    
    
\end{document}
